\section{Public Housing Data Construction}
\label{app:data_construction}

This appendix describes the construction of the public housing dataset used in this analysis, which combines administrative data from the U.S. Department of Housing and Urban Development (HUD) with hand-digitized historical records to create a comprehensive database of public housing projects built between 1930 and 1980.

\subsection{Motivation and Limitations of Existing Datasets}

No single existing dataset provides both precise geographic locations and construction timing for the universe of mid-century public housing projects, necessitating the integration approach described here. The key limitations of available data sources are:

\textbf{Modern HUD Administrative Datasets} contain precise geographic coordinates for public housing projects but lack reliable construction dates, making it impossible to establish treatment timing for historical analysis. While these data excel at capturing current project locations, they provide limited historical information about when projects were built or their original characteristics.

\textbf{Consolidated Directory Dataset (CDD)} provides comprehensive coverage of all public housing projects completed by 1973 along with detailed project characteristics and exact completion dates, but lacks geographic coordinates. This timing information is crucial for causal identification but insufficient without spatial precision.

\textbf{State and Local Funding Gaps} in federal datasets are particularly pronounced for major metropolitan areas. New York City, for example, had extensive state and local funding for public housing that does not appear in federal administrative records, leading to significant undercounting of treatment intensity in precisely the areas where public housing had the largest impact.

To address these limitations, this analysis combines federal administrative records with systematic digitization of historical sources for eight major metropolitan areas. This intensive data collection effort serves two purposes: (1) providing independent verification of the administrative data merge quality, and (2) capturing state and locally-funded projects absent from federal records. The resulting dataset represents what we believe to be the most complete spatial database of mid-century American public housing projects currently available, combining the timing precision of administrative records with the geographic accuracy required for spatial causal identification.

\subsection{Data Sources}

The final public housing dataset integrates four primary data sources:

\subsubsection{HUD Administrative Datasets}

\paragraph{Consolidated Directory Dataset (CDD), 1973} The CDD provides comprehensive information on all public housing projects completed by 1973, including project names, completion dates, unit counts, and basic demographic characteristics. This dataset serves as the backbone for projects outside major metropolitan areas where detailed digitized records are available.

\paragraph{Picture of Subsidized Households (PIC77), 1977} The 1977 PIC dataset, originally compiled by Yana Kucheva, contains detailed household-level demographic information for public housing residents, including racial composition, household size, income, and age structure. Population estimates are derived by multiplying average household size by the number of households, with racial breakdowns calculated using household shares.

\paragraph{HUD Form 951, 1988-1989} The HUD 951 data provides geocoded address information for public housing projects, collected through standardized forms submitted by project managers. These data include precise latitude and longitude coordinates, 1990 census tract identifiers, and detailed address information, processed and geocoded by HUD contractors in 1995-96.

\paragraph{HUD National Geospatial Data Archive (NGDA)} Modern HUD administrative records providing current project locations and characteristics, used primarily for cross-validation and filling gaps in historical records.

\subsubsection{Hand-Digitized Historical Records}

To obtain complete coverage and accurate location data for major metropolitan areas, historical public housing records were digitized from annual reports, FOIA requests, and other primary sources for the following cities:

\paragraph{New York City} Data digitized from the New York City Housing Authority's December 1973 population database, supplemented with initial racial composition data from Max Guennewig-Moenert covering the period through 1971. This provides comprehensive coverage of NYCHA projects with precise location and demographic information.

\paragraph{Chicago} Project information digitized from the Chicago Housing Authority's 1973 Annual Statistical Report, including detailed racial composition data based on family shares and population estimates.

\paragraph{San Francisco} Data compiled from the San Francisco Housing Authority's 1966 and 1973 annual reports, providing project locations defined by street boundaries and completion dates.

\paragraph{Boston} Federal projects digitized from multiple sources including FOIA requests and annual reports, integrated with CDD data for comprehensive coverage.

\paragraph{Washington D.C.} Project data from the National Capital Housing Authority's 1972 annual report, focused on locally-funded projects not captured in federal datasets.

\paragraph{Additional Cities} Supplementary digitization for Baltimore, Los Angeles, and Atlanta using annual reports and administrative records.

\subsection{Geocoding and Location Verification}

\subsubsection{Geocoding Methodology}

Address geocoding was performed using the Google Maps API with the following systematic approach:

\begin{enumerate}
    \item \textbf{Address Parsing}: For projects defined by street boundaries rather than specific addresses, intersection points were created using the first two and last two streets mentioned in project descriptions.
    
    \item \textbf{Multiple Geocoding Attempts}: Each project was geocoded using multiple approaches:
    \begin{itemize}
        \item Full address strings with city and state
        \item Project names with city and state
        \item Parsed intersection coordinates
    \end{itemize}
    
    \item \textbf{Distance Validation}: Distances between different geocoding results were calculated to identify and resolve inconsistencies. Projects with distances exceeding 1-2 kilometers between methods were manually reviewed.
    
    \item \textbf{Midpoint Calculation}: For projects spanning multiple street intersections, coordinates were calculated as the midpoint between boundary points using great circle path calculations.
\end{enumerate}

\subsubsection{Manual Verification and Corrections}

\begin{enumerate}
    \item \textbf{Systematic Review}: All geocoded locations were systematically reviewed using reverse geocoding to verify address accuracy.
    
    \item \textbf{Undergraduate Research Assistant Verification}: A subset of projects with large distances between geocoding methods were manually verified by undergraduate research assistants using historical maps and contemporary sources.
    
    \item \textbf{Manual Coordinate Fixes}: High-profile projects with known historical locations (e.g., Pruitt-Igoe in St. Louis, Brewster in Detroit) were manually assigned precise coordinates based on historical documentation.
    
    \item \textbf{Quality Flags}: Each project includes a source flag indicating whether coordinates came from address geocoding, name geocoding, midpoint calculation, or manual verification.
    
    \item \textbf{Historical Validation}: All projects built before 1949 were cross-referenced against the National Register of Historic Places Continuation Sheet from the U.S. Department of the Interior to ensure comprehensive coverage of early public housing developments.
\end{enumerate}

\subsection{Data Integration and Harmonization}

\subsubsection{CDD-Administrative Data Integration}

The core challenge in combining the CDD dataset (which contains construction timing but no locations) with modern HUD administrative datasets (which contain precise coordinates but no timing) required a systematic project code-based merge strategy.

\paragraph{Project Code Construction and Standardization}
All HUD datasets use a standardized project identification system where project codes follow the format STATE-LOCALITY-PROJECT:
\begin{itemize}
    \item STATE: Two-letter state abbreviation
    \item LOCALITY: Three-digit HUD locality code
    \item PROJECT: Project-specific identifier
\end{itemize}

Project codes were extracted and standardized across all datasets through systematic cleaning procedures including removal of special characters and spaces, conversion to uppercase, and resolution of variant formats.

\paragraph{Multi-Stage Merge Procedure}
The CDD dataset was merged to modern HUD administrative datasets using a hierarchical approach to maximize location coverage:

\begin{enumerate}
    \item \textbf{Primary Match}: Projects were first matched using full project codes (including suffix letters where present) between CDD and PSH datasets (1997, 2000).
    
    \item \textbf{Secondary Match}: Projects unmatched in the primary stage were matched using base project codes (without suffix letters) to PSH datasets, HUD 951, and NGDA data.
    
    \item \textbf{Location Prioritization}: When multiple location sources existed for a single project, coordinates were assigned using the following hierarchy: HUD 951 (highest quality) > PSH 2000 > PSH 1997 > NGDA.
    
    \item \textbf{Manual Corrections}: High-profile projects with known historical locations (e.g., Pruitt-Igoe in St. Louis, Brewster in Detroit) were manually assigned precise coordinates when automated matching failed.
\end{enumerate}

This merge procedure achieved 80.7\% geocoding success for individual projects and 91\% success for total housing units, with 91.6\% success among the top 50 localities by unit count.


\subsubsection{Temporal Harmonization}

Treatment years were assigned based on project completion dates, with projects grouped into decade bins (1930s-1940, 1940s-1950, etc.) to account for:
\begin{itemize}
    \item Uncertainty in exact completion dates
    \item Phased construction over multiple years
    \item Census data availability at decadal intervals
\end{itemize}

Projects with completion dates before 1941 are excluded from the analysis to preserve 1930 and 1940 as pure pre-treatment periods.

\subsubsection{Unit Count Reconciliation}

Total unit counts were reconciled across sources using the following hierarchy:
\begin{enumerate}
    \item Digitized data (when available and verified)
    \item CDD planned units (totunitsplanned)
    \item CDD actual units (totunits)
    \item Administrative dataset unit counts
\end{enumerate}

For projects with construction spanning multiple decades, unit counts were allocated proportionally across treatment periods based on completion timing.

\subsection{Race and Population Data Integration}

\subsubsection{Population Estimation Methodology}

Total population estimates were derived using:
$$\text{Population} = \text{Households} \times \text{Average Household Size}$$

Where average household size comes from the 1977 PIC data (TPERMN variable), with missing values imputed using locality-level means.

\subsubsection{Racial Composition Data}

\paragraph{New York City} Initial racial composition data were merged using a comprehensive manual matching algorithm between digitized project names and the Guennewig-Moenert dataset. Population by race was calculated as:
$$\text{Black Population} = \frac{\text{Black Share}}{100} \times \text{Total Population}$$

\paragraph{Chicago} Racial composition derived from family-level data in the 1973 CHA report:
$$\text{Black Population} = \frac{\text{Black Families}}{\text{Total Families}} \times \text{Total Population}$$

\paragraph{Other Cities} Racial composition data were not systematically available for other digitized cities and are excluded from race-specific analyses.

\subsection{Sample Construction and Quality Metrics}

\subsubsection{Final Sample Characteristics}

The final dataset includes:
\begin{itemize}
    \item 8,016 census tracts with complete neighborhood variables (1930-1990)
    \item Projects with minimum 50 units (increased from 30 units in earlier versions)
    \item Treatment years spanning 1941-1980
    \item Geographic coverage across all U.S. states and territories
\end{itemize}

\subsubsection{Data Quality Assessment}

\paragraph{Geocoding Success Rates}
\begin{itemize}
    \item 80.7\% of projects successfully geocoded
    \item 91\% of total units have valid coordinates
    \item 91.6\% of units in top 50 localities have coordinates
\end{itemize}

\paragraph{Coverage by Data Source}
\begin{table}[h]
\centering
\caption{Public Housing Projects by Data Source}
\begin{tabular}{lcc}
\hline
Data Source & Projects & Units \\
\hline
Digitized Records & 2,847 & 547,231 \\
CDD Administrative & 4,169 & 423,892 \\
Combined (Boston) & 127 & 8,847 \\
\hline
Total & 7,143 & 979,970 \\
\hline
\end{tabular}
\end{table}

\subsubsection{Spatial Distribution}

The dataset provides comprehensive coverage of public housing construction across the United States, with particular strength in major metropolitan areas where digitized records supplement administrative data. Projects are distributed across urban, suburban, and rural contexts, enabling analysis of heterogeneous treatment effects across different geographic settings.

\section{Census Tract Data Construction and Harmonization}
\label{app:census_data}

This section describes the construction of a consistent panel of census tract-level neighborhood characteristics spanning 1930-1990, which serves as the outcome data for analyzing public housing effects.

\subsection{Spatial Standardization to 1950 Census Tract Boundaries}

\subsubsection{Motivation for 1950 Tract Standard}

Census tract boundaries changed substantially across decades as urban areas grew and the Census Bureau refined geographic definitions. To enable longitudinal analysis, all census data from 1930-1990 are harmonized to 1950 census tract boundaries (GISJOIN\_1950) using area-weighted crosswalks. The 1950 standard was chosen because:

\begin{itemize}
    \item 1950 represents the midpoint of the analysis period (1930-1990)
    \item Tract definitions in 1950 provide good coverage of urban areas where public housing was concentrated
    \item Most public housing projects were built after 1940, making 1950 boundaries contemporaneous with treatment
    \item 1950 boundaries balance stability with sufficient geographic resolution for spatial analysis
\end{itemize}

\subsubsection{Area-Weighted Crosswalk Construction}

Geographic crosswalks between each decade's tract boundaries and 1950 tract boundaries were constructed using area-weighted interpolation, adapting the methodology of Eckert et al. (2020):

\begin{enumerate}
    \item \textbf{Spatial Intersection}: For each source year (1930, 1940, 1960, 1970, 1980, 1990), census tract shapefiles are spatially intersected with 1950 tract boundaries to identify overlapping areas.
    
    \item \textbf{Weight Calculation}: For each source tract, weights are calculated as the proportion of the tract's area that falls within each 1950 tract:
    $$w_{ij} = \frac{\text{Area}(\text{Tract}_i \cap \text{Tract}_{1950,j})}{\text{Area}(\text{Tract}_i)}$$
    
    \item \textbf{Weight Normalization}: Weights are renormalized to ensure they sum to 1 for each source tract, accounting for minor geometric inconsistencies in shapefiles.
    
    \item \textbf{Population Reallocation}: Census variables are reallocated to 1950 tract boundaries using these weights:
    $$\text{Population}_{1950,j} = \sum_{i} w_{ij} \times \text{Population}_{i}$$
\end{enumerate}

\subsection{Data Sources and Variable Construction}

\subsubsection{NHGIS Tract-Level Data}

Neighborhood characteristics are compiled from National Historical Geographic Information System (NHGIS) tract-level tabulations covering:

\begin{itemize}
    \item \textbf{Population Demographics}: Total population, population by race (White, Black, other), population by age structure
    \item \textbf{Housing Characteristics}: Total housing units, tenure (owner/renter), housing quality measures, rent levels
    \item \textbf{Economic Indicators}: Occupation distributions, employment status, income measures (where available)
    \item \textbf{Educational Attainment}: School enrollment, educational completion by age groups
    \item \textbf{Geographic Controls}: Central Business District (CBD) indicators, metropolitan area classifications
\end{itemize}

\subsubsection{Full-Count Census Integration}

For 1930 and 1940, tract-level tabulations are supplemented with full-count census microdata processed through enumeration district (ED) to tract crosswalks:

\begin{enumerate}
    \item \textbf{ED-Level Aggregation}: Individual records from full-count census data are collapsed to enumeration district level using household addresses and ED identifiers.
    
    \item \textbf{ED-to-Tract Mapping}: EDs are mapped to 1950 census tracts using Geographic Reference Files (GRF) and area-weighted crosswalks where direct mapping is unavailable.
    
    \item \textbf{Variable Supplementation}: Full-count data provide detailed income measures (constructed from occupational scores using LIDO methodology) and employment variables not available in published tabulations.
\end{enumerate}

\subsection{Key Variable Definitions}

\subsubsection{Demographic Variables}
\begin{itemize}
    \item \textbf{Population by Race}: Consistently defined White and Black population shares across decades, with other racial categories aggregated due to changing definitions
    \item \textbf{Population Density}: Population per square mile using consistent 1950 tract areas
\end{itemize}

\subsubsection{Socioeconomic Variables}
\begin{itemize}
    \item \textbf{Income Measures}: Median family income (deflated to 1950 dollars using CPI), with 1930 income constructed from occupational scores for full-count data
    \item \textbf{Education}: Share of adults with high school completion, using age-appropriate definitions for each decade
    \item \textbf{Employment}: Labor force participation rates, unemployment rates
\end{itemize}

\subsubsection{Housing Variables}
\begin{itemize}
    \item \textbf{Tenure}: Homeownership rates, rental shares
    \item \textbf{Housing Quality}: Share of housing lacking plumbing, overcrowding measures
    \item \textbf{Rent Levels}: Median contract rent (in constant 1950 dollars)
\end{itemize}

\subsection{Sample Construction and Quality}

\subsubsection{Final Sample Characteristics}

The final census tract panel includes:
\begin{itemize}
    \item 8,016 census tracts with complete variables across 1930-1990
    \item Geographic coverage across all U.S. states, with concentration in metropolitan areas
    \item Balanced panel structure with consistent tract definitions over time
\end{itemize}

\subsubsection{Data Quality Considerations}

\begin{enumerate}
    \item \textbf{Boundary Change Adjustments}: Area-weighted crosswalks account for the majority of boundary changes, though some measurement error is inevitable in areas with substantial tract reorganization.
    
    \item \textbf{Variable Consistency}: Census variable definitions evolved across decades; variables are harmonized to maintain comparability while acknowledging definitional limitations.
    
    \item \textbf{Missing Data Treatment}: Tracts with incomplete data in any decade are excluded to maintain panel balance, potentially biasing toward more established urban areas.
\end{enumerate}

\subsection{Additional Control Variables}

\subsubsection{Central Business District Identification}

CBD indicators are constructed using the Central Business District flag available in 1980 NHGIS census tract data and concorded to 1950 tract boundaries using the same area-weighted crosswalk methodology. These indicators capture downtown urban cores and are used as controls for pre-existing neighborhood characteristics that may have influenced public housing site selection.

\subsubsection{HOLC Redlining Maps}

Historical redlining data from the Home Owners' Loan Corporation (HOLC) are integrated using digitized maps from the Mapping Inequality project. HOLC grades (A-D, with D representing "hazardous" areas) are assigned to 1950 census tracts using spatial overlays:

\begin{enumerate}
    \item HOLC polygon boundaries are spatially intersected with 1950 census tract boundaries
    \item Tracts are classified as "redlined" if more than 80\% of their area falls within HOLC grade D zones
    \item New York City boroughs (Manhattan, Brooklyn, Bronx, Queens, Staten Island) are consolidated into a single metropolitan area for consistency
\end{enumerate}

\subsubsection{Urban Renewal Areas}

Urban renewal project boundaries are compiled from the "Renewing Inequality" dataset, providing comprehensive coverage of urban renewal activities from the 1950s-1980s. The data processing involves:

\begin{enumerate}
    \item Overlapping urban renewal polygons are unioned to create consolidated urban renewal areas
    \item Census tracts are classified as affected by urban renewal based on spatial intersection with these consolidated areas
    \item Tract-level indicators capture both the presence and intensity of urban renewal exposure
\end{enumerate}

\subsubsection{Highway Proximity}

Interstate highway proximity measures are constructed using highway location data from Weiwu (2024), calculating:

\begin{itemize}
    \item Distance from each tract centroid to the nearest interstate highway (in kilometers and miles)
    \item Binary indicators for highway proximity within 1km, 2km, 1 mile, and 2 mile buffers
    \item All distance calculations performed using US Albers Equal Area projection (EPSG:5070) for accurate measurements
\end{itemize}

\subsection{Historical Median Calculation Methodology}

For census years where median values are reported as categorical ranges rather than continuous measures (including full-count data for 1930 and 1940, and NHGIS tabulations for 1960 and 1970), medians are approximated using the following procedure:

\begin{enumerate}
    \item \textbf{Frequency Distribution Construction}: Census tabulations providing counts within income/value ranges (e.g., \$1,000-\$1,500, \$1,500-\$2,000) are converted to frequency distributions.
    
    \item \textbf{Cumulative Frequency Calculation}: Cumulative frequencies are calculated across ordered ranges to identify the median position (50th percentile).
    
    \item \textbf{Median Range Identification}: The range containing the median position is identified as the first range where cumulative frequency exceeds half the total population.
    
    \item \textbf{Conservative Approximation}: The lower bound of the median-containing range is used as the median estimate, providing a conservative approximation that avoids interpolation assumptions about within-range distributions.
    
    \item \textbf{Midpoint Conversion}: For final analysis, identified median ranges are converted to midpoint values (e.g., \$1,000-\$1,500 range becomes \$1,250) except for open-ended upper categories where reasonable upper bounds are assigned.
\end{enumerate}

This methodology is implemented through a standardized function (\texttt{calculate\_median\_from\_census}) applied consistently across multiple variables including income, housing values, and rental costs for both full-count census data (1930, 1940) and NHGIS tabulations (1960, 1970).

\subsection{Data Limitations and Considerations}

\subsubsection{Measurement Error}

\begin{enumerate}
    \item \textbf{Geocoding Precision}: While systematic verification procedures were employed, some measurement error in project coordinates is inevitable, particularly for projects defined by street boundaries rather than specific addresses.
    
    \item \textbf{Temporal Precision}: Exact completion dates are sometimes approximated, particularly for projects with phased construction.
    
    \item \textbf{Unit Count Reconciliation}: Minor discrepancies exist between administrative sources regarding total unit counts.
\end{enumerate}

\subsubsection{Coverage Limitations}

\begin{enumerate}
    \item \textbf{Scattered-Site Projects}: Projects explicitly described as "scattered sites" were excluded due to inability to assign precise locations.
    
    \item \textbf{Racial Composition Data}: Comprehensive racial composition data are available only for New York City and Chicago projects.
    
    \item \textbf{Rural Projects}: Coverage may be less complete for rural areas not well-represented in administrative datasets.
\end{enumerate}

\subsubsection{Data Integration Challenges}

The integration of multiple data sources with different collection methodologies, time periods, and geographic precision requirements necessitated extensive harmonization efforts. While systematic procedures were employed to ensure consistency, researchers should be aware that this dataset represents a reconstruction of historical records with the attendant limitations of such efforts.

\subsection{Replication and Data Availability}

All data processing scripts and digitization templates are available in the project repository. The Google Maps API key used for geocoding would need to be replaced by replicators. Raw digitized data files and intermediate processing outputs are preserved to enable verification and extension of the dataset.