\section{High-Resolution Land Value Analysis: Chicago Public Housing Effects}
\label{app:chicago_land_values}

This appendix presents a supplementary spatial analysis of public housing effects on land values using unprecedented high-resolution data from Chicago. This analysis complements the main tract-level results by examining effects at a finer spatial scale, providing additional evidence on the geographic scope and magnitude of public housing impacts.

\subsection{Data and Sample}

\subsubsection{Ahlfeldt-McMillen Land Value Dataset}

The land value analysis utilizes the Ahlfeldt and McMillen (2019) dataset, which provides comprehensive land value estimates for Chicago at extraordinary spatial resolution.\footnote{Ahlfeldt, Gabriel M., and Daniel P. McMillen. ``The properties of property: Land use and land values.'' \emph{Journal of Urban Economics} 110 (2019): 20-33.} Key features include:

\begin{itemize}
    \item \textbf{Spatial Resolution}: 300×300 foot grid cells providing fine-grained geographic coverage
    \item \textbf{Temporal Coverage}: Annual observations from 1913-2009
    \item \textbf{Geographic Scope}: Comprehensive coverage of Cook County, Illinois
    \item \textbf{Outcome Variable}: Log land value per square foot (llv) derived from property tax assessments and sales data
\end{itemize}

\subsubsection{Chicago Public Housing Projects}

The public housing sample includes projects meeting the following criteria:
\begin{itemize}
    \item \textbf{Location}: Chicago, Illinois (Cook County)
    \item \textbf{Construction Period}: Completed 1939-1971
    \item \textbf{Project Size}: Minimum 50 housing units
    \item \textbf{Geocoding}: Successfully geocoded with precise latitude and longitude coordinates
\end{itemize}

This yields a final sample of public housing projects providing variation in construction timing, project size, and geographic location across Chicago's diverse neighborhoods.

\subsection{Methodology}

\subsubsection{Spatial Ring Assignment}

The analysis employs a spatial difference-in-differences design comparing land value changes across concentric rings around public housing projects:

\begin{itemize}
    \item \textbf{Treatment Rings}: 0-200m, 200-400m, 400-600m from project locations
    \item \textbf{Control Ring}: 800-1000m from project locations  
    \item \textbf{Distance Calculations}: Euclidean distances using NAD83 Illinois East Zone projection for accuracy
\end{itemize}

This ring structure allows identification of the spatial gradient of effects while avoiding spillover contamination between treatment and control areas.

\subsubsection{Stacked Difference-in-Differences Design}

The estimation strategy follows a stacked difference-in-differences approach accommodating multiple treatment timing:\footnote{Wing, Coady, Kevin Simon, and Roberto A. Bielby. ``Designing difference in difference studies: best practices for public health policy research.'' \emph{Annual Review of Public Health} 44 (2023): 473-495.}

\begin{enumerate}
    \item \textbf{Stacked Structure}: Each public housing project creates a separate ``sub-experiment'' comparing spatial rings around that project
    \item \textbf{Wing (2024) Weights}: Observations are weighted to ensure proper treatment effect estimation in stacked designs
    \item \textbf{Event Study Framework}: Effects are estimated relative to the year before project completion (t=-1)
\end{enumerate}

\subsubsection{Regression Specification}

The baseline specification estimates:

\begin{align}
\text{llv}_{igt} = \sum_{k \neq -1} \sum_{r} \beta_{kr} \mathbf{1}[\text{event\_time}_{it} = k] \times \mathbf{1}[\text{location\_type}_{ig} = r] + \gamma_{igt} + \epsilon_{igt}
\end{align}

where $i$ indexes grid cells, $g$ indexes projects, $t$ indexes years, $k$ indexes event time, and $r$ indexes spatial rings. The fixed effects $\gamma_{igt}$ include:

\begin{itemize}
    \item \textbf{Project×Year}: $\alpha_{gt}$ controls for project-specific time trends
    \item \textbf{Project×Location Type}: $\mu_{gr}$ controls for baseline differences across spatial rings within each project
    \item \textbf{Project×Urban Renewal}: $\delta_{g} \times \text{UR}_{ig}$ allows heterogeneous effects by urban renewal exposure  
    \item \textbf{Project×Highway Proximity}: $\theta_{g} \times \text{Highway}_{ig}$ controls for transportation infrastructure
\end{itemize}

The crucial innovation is the \textbf{Project×Location Type} fixed effects ($\mu_{gr}$), which ensure identification comes from within-project spatial-temporal variation rather than cross-project comparisons. This specification preserves the spatial gradient necessary for identification while controlling for all time-invariant characteristics that vary across projects and spatial rings.

\subsection{Results}

Table~\ref{tab:chicago_all_rings_main} presents the main results from the spatial land value analysis. The baseline specification demonstrates clean identification and reveals positive effects on land values near public housing projects.

\subsubsection{Pre-Treatment Trends}

The specification with proper project×location type fixed effects demonstrates clean pre-treatment trends across all spatial rings:
\begin{itemize}
    \item Event times t=-3 and t=-2 show small, statistically insignificant coefficients across all rings
    \item Coefficients range from -0.02 to +0.08, indicating no systematic pre-existing trends
    \item This strongly supports the parallel trends assumption underlying spatial difference-in-differences identification
\end{itemize}

\subsubsection{Treatment Effects}

The post-treatment period reveals significant positive effects on land values, concentrated in the closest rings:

\begin{itemize}
    \item \textbf{0-200m Ring}: Significant positive effects emerge at t=0 (+7.1%, p<0.10) and strengthen over time, reaching +15.0% by t=4 (p<0.10)
    \item \textbf{200-400m Ring}: Modest positive effects that build gradually, with some periods showing statistical significance
    \item \textbf{400-600m and 600-800m Rings}: Smaller positive effects that are generally not statistically significant
\end{itemize}

The spatial pattern shows a clear distance gradient: effects are strongest in the immediate vicinity of public housing projects and decay with distance. This suggests positive spillovers from public housing construction that are geographically localized, consistent with complementary infrastructure investment or agglomeration benefits.

\subsubsection{Heterogeneity by Project Size}

Table~\ref{tab:chicago_hetero_size_median} examines how public housing effects vary by project scale using a median split (≤202 vs >202 units), comparing 55 smaller projects to 18 larger developments. This analysis reveals substantial heterogeneity in neighborhood land value impacts:

\begin{itemize}
    \item \textbf{Smaller Projects} (Column 1): Demonstrate clean pretrends followed by large positive effects that build steadily over time. Effects reach +32.5% at t=5 (p<0.01) in the closest ring, with significant positive spillovers extending to the 200-400m ring (+20.6% at t=5, p<0.05). The temporal pattern shows consistent strengthening of positive effects.
    
    \item \textbf{Larger Projects} (Column 2): Show relatively stable patterns through t=4, but negative effects emerge at t=5 (-11% to -18% across rings). The concentration of negative effects at the longest horizon raises questions about whether these reflect genuine treatment effects or estimation artifacts from diminishing sample size at longer lags.
\end{itemize}

The substantial heterogeneity by project scale likely reflects multiple mechanisms. Smaller projects may have integrated into existing neighborhood fabric with minimal disruption, potentially enhancing property values through improved housing quality. Larger projects, while providing more units, may have reached a threshold where institutional scale disrupted neighborhood dynamics or created visible concentrations that affected market perceptions. However, project size may proxy for other factors including site selection patterns, construction quality, and coordination with urban renewal programs, all of which varied systematically with project scale during this era.

While these patterns suggest project characteristics matter for neighborhood impacts, several limitations caution against strong conclusions: the imbalanced sample split, concentration of negative effects at t=5, and inability to rule out confounding factors related to site selection and project design.

\subsection{Interpretation and Limitations}

\subsubsection{Consistency with Main Results}

The high-resolution land value analysis provides important nuance to the tract-level findings presented in the main analysis. The positive effects on land values for smaller projects, combined with negative effects for large projects, help explain the mixed results often found in public housing research. The project size heterogeneity suggests that public housing impacts operated through different mechanisms depending on scale: smaller projects facilitated positive neighborhood transformation through localized spillovers, while large projects generated negative externalities through concentration effects.

\subsubsection{Methodological Contributions}

This analysis demonstrates the value of high-resolution spatial data for understanding public housing effects:
\begin{itemize}
    \item The 300×300 foot resolution reveals spatial gradients invisible in tract-level analysis
    \item Clean identification of distance-based effects provides evidence on the geographic scope of impacts
    \item Stacked difference-in-differences design with proper weighting ensures valid causal inference across multiple treatment timing
\end{itemize}

\subsubsection{Limitations}

Several limitations should be acknowledged:
\begin{itemize}
    \item \textbf{Single City Analysis}: Results are specific to Chicago and may not generalize to other metropolitan areas
    \item \textbf{Land Value Focus}: Effects on property values may not capture the full range of neighborhood impacts
    \item \textbf{Sample Period}: Analysis covers 1913-2009 but most projects were built 1939-1971, limiting long-run perspective
\end{itemize}

\subsection{Conclusion}

The high-resolution land value analysis provides suggestive evidence that public housing effects vary by project scale. Smaller developments show consistent positive spillovers reaching +33\% in the immediate vicinity, with effects strengthening over time. Larger developments show more mixed patterns, though interpretation is complicated by effects concentrating at the longest time horizon.

These findings provide historical evidence from mid-20th century Chicago supporting smaller-scale approaches to subsidized housing development. The positive effects of projects under 200 units align with modern scattered-site strategies that aim to integrate affordable housing into existing neighborhoods. However, these results reflect Chicago's specific policy environment during 1939-1971, and contemporary policymakers should view these findings as informing rather than determining modern policy choices, given differences in current funding, design, and management practices.

The analysis serves as a methodological proof-of-concept for using high-resolution spatial data to understand heterogeneous policy effects, while demonstrating that average treatment effects can mask substantial variation in outcomes depending on implementation characteristics.

The analysis demonstrates the value of high-resolution spatial data and proper within-project identification for understanding the nuanced effects of place-based policies on American neighborhoods.