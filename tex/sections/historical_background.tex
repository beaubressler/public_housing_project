\section{Historical Background: The U.S. Public Housing Program}\label{sec:background}



\subsection{Program Structure and Evolution}
The scope and focus of the public housing program evolved substantially over several phases.

Public housing in the United States emerged in earnest during the Great Depression. The first public housing projects were built by the Public Works Administration (PWA) as part of New Deal efforts to alleviate urban housing shortages and provide economic stimulus. The New Deal public housing program ultimately produced only about 20,000 housing units, as court challenges limited the ability of the PWA to acquire land for public housing (\cite{jacksonCrabgrassFrontierSuburbanization2006}).

The public housing program was greatly expanded and decentralized with the passage of the Housing Act of 1937, which encouraged the creation of local Public Housing Authorities (PHAs) to construct and manage public housing with federal funding. In particular, federal grants would pay the difference between the cost of operating the projects and what PHAs would receive from tenants. PHAs would be responsible for site selection and operation of the projects. The stated public purpose of the 1937 Act was the clearing and redevelopment of urban "slums", rather than the growth of housing supply, based on beliefs at the time that these neighborhoods reduced nearby property values and caused poor behavior and health among their residents (\cite{meyersonPoliticsPlanningPublic1955}). Over 170,000 housing units were built under the Housing Act of 1937 (\cite{schwartzHousingPolicyUnited2021}), with almost 90\% built on slum sites. Contrary to later perceptions of public housing, these early projects were largely not intended to house the very poorest, but instead targeted the "submerged" middle class (\cite{friedmanGovernmentSlumHousing1967}), with PHAs often carefully screening tenants.

During World War II, public housing construction stalled, with construction largely focusing on housing for defense workers. However, the post-war period brought renewed attention to urban conditions, particularly acute housing shortages for African Americans in northern cities following the Great Migration. The 1949 Housing Act authorized funding for an additional 810,000 units of public housing within a broader framework of urban renewal (\cite{meyersonPoliticsPlanningPublic1955}). 

This legislation marked a significant shift in the program's scope and ambition, expanding public housing as a tool for slum clearance, an expansion of low-income housing, as well as a potential destination for those who would be displaced by the urban renewal program. Indeed, the Act articulated an expansive vision: ``the general welfare and security of the Nation and the health and living standards of its people require housing production and related community development sufficient to remedy the serious housing shortage, the elimination of substandard and other inadequate housing through the clearance of slums and blighted areas, and the realization as soon as feasible of the goal of a decent home and a suitable living environment for every American family.'' 

The implementation reality, however, fell far short of these ambitious goals. In Chicago, for example, the city's public housing program became a focal point for racial and political conflict. The Chicago Housing Authority initially planned to build projects throughout the city, but intense opposition from white neighborhoods led to the concentration of projects in predominantly Black areas on the South and West sides (\cite{hirschMakingSecondGhetto1998}). Similar patterns emerged in other major cities. High-profile failures of the public housing program like the Pruitt-Igoe Homes in St. Louis, built in 1954 and demolished by 1972, became symbols of an urban policy failure that influenced national policy debates.

In response to growing criticism, federal housing policy shifted away from direct provision of public housing. In 1973, President Nixon declared a moratorium on subsidies for traditional public housing (\cite{bloomPublicHousingMyths2015}), and the 1970s and 1980s saw the move towards demand-side subsidies like the Section 8 Housing Choice Vouchers program and subsidies for private provision of affordable housing like the Low Income Housing Tax Credit.

By the early 1990s, the public housing stock faced serious challenges: physical deterioration due to deferred maintenance, extreme concentration of poverty as working families moved out, and in some cases, rampant crime and drug activity. These struggles laid the groundwork for the HOPE VI program, launched in 1992, which provided federal funding to demolish distressed projects or transform them into mixed-income developments. 


\subsection{Race, Segregation, and Public Housing}

Race and segregation emerged as central and contentious issues throughout the program's history. The housing projects built by the Public Works Administration followed a ``neighborhood composition rule'', which determined the racial composition of a project based on the existing racial composition of the neighborhood in which it was built. This policy, ostensibly designed to maintain neighborhood stability, in practice reinforced existing patterns of segregation. In some cases, however, segregated public housing projects were constructed in previously integrated neighborhoods, effectively creating segregation where it previously did not exist (\cite{rothsteinColorLawForgotten2018}). For example, the Techwood Homes built in Atlanta in 1936, was an all-white project built over a previously racially integrated low-income neighborhood. 

Many large public housing authorities often continued to follow an explicit neighborhood composition rule into the 1950s. However, even after segregation was formally banned, public housing projects still remained highly racially segregated(\cite{kuchevaSubsidizedHousingConcentration2013}). Contemporaneous debates among policy-makers about the role of public housing in promoting or fighting racial segregation reveal that officials were well aware of the program's potential to either exacerbate or ameliorate residential segregation patterns (\cite{hirschContainmentHomeFront2000}).

This historical context is crucial for understanding the neighborhood effects of public housing construction. The systematic targeting of disadvantaged areas means that any analysis of neighborhood impacts must account for selection bias in where projects were built.


%\subsection{Implications for }
%This historical context has crucial implications for understanding the neighborhood effects of public housing construction. First, the systematic targeting of already disadvantaged areas means that public housing was not randomly distributed across neighborhoods but was concentrated in areas that were likely on different trajectories even before construction began. This selection bias makes it essential to use causal identification strategies, like the matched difference-in-differences approach employed in this paper, to separate the effects of public housing from pre-existing neighborhood trends.
%Second, the racial dynamics of site selection suggest that public housing may have accelerated processes of neighborhood racial transition that were already underway. Rather than causing white flight per se, public housing construction may have served as a signal of neighborhood racial change that influenced private market decisions and accelerated existing demographic trends.
%Third, the political economy of site selection reveals how public housing became embedded within broader systems of racial and economic segregation rather than serving as a tool to break down those barriers. The concentration of projects in already disadvantaged areas meant that public housing reinforced existing patterns of disadvantage rather than creating opportunities for spatial mobility and integration.
%Finally, the evolution from the "deserving poor" focus of early public housing to the more concentrated poverty of later periods suggests that the neighborhood effects documented in this paper may reflect not just the direct impacts of public housing construction, but also the changing social and economic profile of public housing residents over time. Understanding these dynamics is crucial for interpreting both the site selection patterns and neighborhood effects that form the core of this analysis.