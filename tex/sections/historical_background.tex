\section{Historical Background: The U.S. Public Housing Program}\label{sec:background}

Public housing in the United States emerged during the Great Depression as part of New Deal efforts to alleviate urban housing shortages and provide economic stimulus.
The Public Works Administration built the first projects, producing about 20,000 units before court challenges over land acquisition limited its scope (\cite{jacksonCrabgrassFrontierSuburbanization1985}).

The program was greatly expanded and decentralized with the passage of the Housing Act of 1937, which encouraged the creation of local Public Housing Authorities (PHAs) to construct and manage public housing with federal funding.
In particular, federal grants would pay the difference between the cost of operating the projects and what PHAs would receive from tenants, while PHAs would be responsible for site selection and operation of the projects.
The Act's stated purpose was slum clearance and redevelopment rather than explicitly expanding housing supply, reflecting contemporary beliefs that deteriorated neighborhoods reduced nearby property values and caused poor health and behavior among residents (\cite{meyersonPoliticsPlanningPublic1955}).
Over 170,000 housing units were built under the Housing Act of 1937, with nearly 90\% built on former slum sites (\cite{schwartzHousingPolicyUnited2021}). 
%Contrary to later perceptions of public housing, these early projects were largely not intended to house the very poorest, but instead targeted the "submerged" middle class (\cite{friedmanGovernmentSlumHousing1967}), with PHAs often carefully screening tenants.

World War II temporarily shifted focus to defense worker housing, but the post-war period brought renewed attention to urban conditions.  %TODO could have another sentence about urban conditions in post-war period, see Lavoice (2024)
The 1949 Housing Act authorized funding for an additional 810,000 units of public housing within a broader program of urban renewal (\cite{meyersonPoliticsPlanningPublic1955}). 
This legislation marked a significant shift in the program's scope and ambition, expanding public housing as a tool for slum clearance, a source of low-income housing, as well as a potential destination for those who would be displaced by the urban renewal program.

The ambitious goals of the 1949 Act, however, quickly met political, fiscal, and social obstacles. 
Several developments in the 1950s and 1960s transformed public housing into a deeply controversial policy.

First, as described in Section \ref{sec:race}, racial dynamics in site selection generated intense political conflict.
Second, the tenant composition shifted dramatically over time. Early projects had housed working families and the "deserving poor," but by the 1960s they increasingly concentrated the most disadvantaged households.
The growth of suburban home ownership enabled working-class families to leave public housing, while authorities tightened income eligibility limits.
Federal regulations also required PHAs to prioritize the neediest applicants, concentrating poverty within projects (\cite{schwartzHousingPolicyUnited2021}).
Third, design and construction problems became increasingly apparent.
To avoid competing with private housing, and due to mandated construction cost limits, projects were built to be deliberately austere and inexpensive, making them prone to rapid deterioration.
The increasingly poor tenant base combined with inadequate funding led to widespread disrepair and maintenance backlogs by the 1960s and 1970s (\cite{schwartzHousingPolicyUnited2021}).
High-profile disasters like St. Louis's Pruitt-Igoe Homes, built in 1954 and demolished in 1972, became national symbols of urban policy failure.
Originally hailed as a model development, Pruitt-Igoe quickly descended into crime, vandalism, and abandonment as maintenance funding dried up and middle-class residents moved out (\cite{bristolPruittIgoeMyth1991}). 
Other critics pointed to the design of the projects themselves, arguing that the modernist high-rise towers and superblocks created alienating environments that fostered social problems (\cite{jacobsDeathLifeGreat1961}).

In response to mounting criticism, federal housing policy shifted away from direct provision of public housing. A 1971 report by the Nixon Administration wrote that "drab, monolithic housing projects, largely segregated...still stand in our cities as prisons of the poor" (\cite{orlebekeEvolutionLowIncomeHousing2000}).
In 1973, President Nixon declared a moratorium on subsidies for traditional public housing.
%Section 8 Housing Choice Vouchers program and subsidies for private provision of affordable housing like the Low Income Housing Tax Credit.
By the early 1990s, the public housing stock faced serious challenges: physical deterioration due to deferred maintenance, extreme concentration of poverty as working families moved out, and in some cases, rampant crime and drug activity.
These struggles laid the groundwork for the HOPE VI program, launched in 1992, which provided federal funding to demolish distressed projects or transform them into mixed-income developments. 

\subsection{Race, Segregation, and Public Housing}\label{sec:race}

Race and segregation emerged as central and contentious issues throughout the program's history. 
From the outset, the Public Works Administration followed a so-called ``neighborhood composition rule'', formally segregating projects according to the demographics of the neighborhoods in which they were built.
While ostensibly a neutral policy intended to maintain neighborhood ``stability'', in practice the rule entrenched segregation.
Prominent accounts, such as \textcite{rothsteinColorLawForgotten2017}, argue that public housing not only reinforced existing segregation, but actively created it by building segregated projects in previously integrated neighborhoods.
A well-known example is Atlanta’s Techwood Homes, constructed in 1936 as an all-white development, which displaced a previously integrated low-income neighborhood (\cite{rothsteinColorLawForgotten2017}).
Many large public housing authorities continued to follow an explicit neighborhood composition rule until explicit racial segregation was banned in 1954.
Yet even after this legal shift, projects remained highly segregated in practice (\cite{bickfordSegregationSecondGhetto1991}).

Historical case studies suggest that issues around race and segregation also shaped site selection decisions, and contemporaneous debates among policy-makers about the role of public housing in promoting or fighting racial segregation reveal that officials were well aware of the program's potential to either exacerbate or ameliorate residential segregation patterns (\cite{hirschContainmentHomeFront2000}).
In Chicago, the Housing Authority initially planned to build projects throughout the city, but fierce opposition from white neighborhoods led to the concentration of projects in predominantly Black areas on the South and West Sides (\cite{meyersonPoliticsPlanningPublic1955, hirschMakingSecondGhetto1998}).
Similar battles played out in other major cities, with white residents and politicians successfully blocking projects in their neighborhoods.
I explore a particular example of this dynamic in Philadelphia, comparing proposed-but-not-selected areas to public housing locations in Appendix \ref{sec:philadelphia_site_selection_appendix}.
Building on this historical evidence, Section \ref{sec:site_selection_where} tests whether these dynamics held systematically nationwide.
