\section{Historical Background: The U.S. Public Housing Program}\label{sec:background}

Public housing in the United States began during the Great Depression as part of New Deal efforts to reduce urban housing shortages and stimulate the economy. The Public Works Administration built the first projects, producing about 20,000 units before court challenges over land acquisition limited its scope (\cite{jacksonCrabgrassFrontierSuburbanization1985}).

The program was greatly expanded and decentralized with the passage of the Housing Act of 1937, which promoted the creation of local Public Housing Authorities (PHAs) to build and oversee public housing with federal funds. Specifically, federal grants covered the difference between operating costs and tenant revenue, while PHAs were responsible for site selection and project management. The 1937 Act's stated purpose was slum clearance and redevelopment rather than expanding housing supply, which reflected contemporary beliefs that ``slum'' neighborhoods reduced nearby property values and caused poor health and behavior among residents (\cite{meyersonPoliticsPlanningPublic1955}).
 170,000 housing units were constructed under this act, with nearly 90\% built on former slum sites (\cite{schwartzHousingPolicyUnited2021}). 
%Contrary to later perceptions of public housing, these early projects were primarily not intended to house the very poorest, but instead targeted the "submerged" middle class (\cite{friedmanGovernmentSlumHousing1967}), with PHAs often carefully screening tenants.

The program was greatly expanded as part of a broad urban policy agenda following World War II.
%World War II temporarily shifted focus to defense worker housing, but the post-war period brought renewed attention to urban conditions.
Many Northern cities experienced severe housing shortages, which were especially acute for African Americans, while white flight, suburbanization, and urban ``blight'' challenged cities. These challenges generated a large political coalition for urban redevelopment policy, culminating in the Housing Act of 1949. (\cite{vonhoffmanStudyContradictionsOrigins2000}). Title III of the 1949 Act provided funding for an additional 810,000 units of public housing and marked a major shift in the program's scope and goals, expanding public housing as a tool for slum clearance, a source of low-income housing, and a potential destination for those displaced by the Urban Renewal slum clearance program, which was funded by Title I of the Act.

The ambitious goals of the 1949 Act, however, quickly met political, fiscal, and social obstacles. 
Several developments in the 1950s and 1960s transformed public housing into a deeply controversial policy.

First, as detailed in Section \ref{sec:race}, racial dynamics in site selection caused intense political conflicts. Second, the composition of public housing tenants changed significantly over time. Early projects primarily housed working families and the ``deserving poor,'' but by the 1960s, they increasingly housed the poorest and most disadvantaged households. The rise of suburban homeownership allowed working-class families to leave public housing, while authorities tightened income eligibility limits. Federal regulations also mandated that PHAs prioritize the neediest applicants, resulting in the concentration of poverty within projects (\cite{schwartzHousingPolicyUnited2021}).
Third, design and construction issues became more evident. To avoid competing with private housing and to meet mandated construction cost limits, projects were intentionally made austere and inexpensive, making them susceptible to rapid deterioration (\cite{schwartzHousingPolicyUnited2021}). Many public housing authorities struggled to fund maintenance as middle-class tenants exited the projects. Furthermore, the 1969 Brooke Amendment capped tenant rents at 25\% of income, further limiting PHA revenues (\cite{huntPublicHousingUrban2018}).
These challenges are illustrated by well-known public housing failures, such as St. Louis's Pruitt-Igoe Homes, built in 1954. Initially praised as a model development, Pruitt-Igoe quickly faced issues like crime, vandalism, and abandonment as maintenance funding dried up and middle-class residents moved out (\cite{bristolPruittIgoeMyth1991}). As a result of these issues, the complex was ultimately demolished beginning in 1972.
%Other critics pointed to the design of the projects themselves, arguing that the modernist high-rise towers and superblocks created alienating environments that fostered social problems (\cite{jacobsDeathLifeGreat1961}).

In response to mounting criticism, American housing policy has shifted away from direct public housing provision. A 1971 report by the Nixon Administration wrote that ``drab, monolithic housing projects, largely segregated...still stand in our cities as prisons of the poor'' (\cite{orlebekeEvolutionLowIncomeHousing2000}).
In 1973, President Nixon declared a moratorium on subsidies for traditional public housing.
%Section 8 Housing Choice Vouchers program and subsidies for private provision of affordable housing, like the Low Income Housing Tax Credit.
By the early 1990s, the public housing stock faced serious challenges: physical deterioration due to deferred maintenance, extreme concentration of poverty as working families moved out, and, in some cases, rampant crime and gang activity.
The federal response to these challenges was the HOPE VI program, launched in 1992, which provided federal funding to demolish distressed projects or transform them into mixed-income developments. The Faircloth Amendment of 1998 further limited public housing growth by capping the total number of units PHAs could operate at their 1999 levels.

\subsection{Race, Segregation, and Public Housing}\label{sec:race}

Race and segregation emerged as central and contentious issues throughout the program's history. 
From the outset, the Public Works Administration followed a so-called ``neighborhood composition rule,'' formally segregating projects based on the demographics of the neighborhoods where they were built. While ostensibly a neutral policy intended to maintain neighborhood ``stability'', in practice the rule solidified segregation. Notable accounts, such as \textcite{rothsteinColorLawForgotten2017}, contend that public housing not only reinforced existing segregation but also actively created it by constructing segregated projects in previously integrated neighborhoods. A well-known example is Atlanta’s Techwood Homes, built in 1936 as an all-white development, which displaced a previously integrated low-income neighborhood (\cite{rothsteinColorLawForgotten2017}). Many large public housing authorities continued to follow an explicit neighborhood-composition rule well into the 1960s.\footnote{Although legal challenges began in the mid-1950s, including the U.S. Supreme Court's 1954 decision to let stand a California ruling striking down San Francisco's segregated projects, federal law did not formally prohibit race-based segregation in public housing until the Fair Housing Act of 1968.} However, even after these legal changes, projects remained highly segregated in practice (\cite{bickfordSegregationSecondGhetto1991}).

Historical case studies show that issues related to race and segregation also influenced site selection decisions. Additionally, contemporaneous debates among policymakers reveal that they were fully aware of the program's potential to either worsen or improve residential segregation patterns (\cite{hirschContainmentHomeFront2000}).
In Chicago, the Housing Authority initially planned to build projects throughout the city, but strong opposition from white neighborhoods led to the concentration of projects in predominantly Black areas on the South and West Sides (\cite{meyersonPoliticsPlanningPublic1955, hirschMakingSecondGhetto1998}).
Similar conflicts occurred in other major cities, with white residents and politicians successfully blocking projects in their neighborhoods. I examine a specific example of this dynamic in Philadelphia, comparing proposed-but-not-selected areas to public housing locations in Section \ref{sec:philadelphia_site_selection}. Building on this historical evidence, Section \ref{sec:site_selection_where} tests whether these racial site selection dynamics held systematically nationwide.


\subsection{Motivating Example: Hunters Point in San Francisco}\label{sec:hunters_point}

The history of Bayview–Hunters Point in San Francisco provides a clear example of how public housing development can dramatically change a neighborhood over the long run.
Before World War II, the area was a racially mixed district on the city's periphery. 
During the war, the federal government built thousands of temporary units there to house defense workers, leading to an initial influx of African American families. % CITE 
The history of Bayview–Hunters Point in San Francisco provides a clear example of how public housing construction can change a neighborhood's long-run trajectory. 
Before World War II, the area was a racially mixed neighborhood on the city's periphery. 
During the war, the federal government built thousands of temporary housing units there to accommodate defense workers, causing an initial influx of African American families. After the war, instead of dismantling these temporary units, local and federal officials converted many of them into permanent public housing. Throughout the 1950s and 1960s, the San Francisco Housing Authority replaced and expanded these units into large-scale projects and directed Black families into these developments. This expansion coincided with a quick demographic shift: as the projects grew, the neighborhood stopped being a mixed-income area; white and Asian households left, and the community became increasingly Black and segregated. Figures \ref{fig:hunters_point_blkshr} and \ref{fig:hunters_point_whitepop} illustrate the racial transition of these neighborhoods along with the location of public housing projects, illustrating the sharp rise in the Black population share and the decline in the white population from 1940 to 1970. The rest of this paper tests whether the patterns seen in Hunters Point were systematic across the country.
After the war, rather than dismantling this temporary stock, local and federal officials converted much of it into permanent public housing. 
Throughout the 1950s and 1960s, the San Francisco Housing Authority replaced and expanded these units into large-scale projects and redirected Black families into them.

This expansion coincided with a rapid demographic shift: As the projects grew, the neighborhood ceased to be a mixed-income area; white and Asian households left, and the community became increasingly Black and segregated. 
Figure \ref{fig:hunters_point} illustrates these changes, showing the sharp increase in the Black population share and declines in the white population from 1950 to 1970. The remainder of this paper tests whether the patterns observed in Hunters Point were systematic nationwide.
