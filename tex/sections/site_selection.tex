\section{Site Selection}\label{sec:site_selection}

In this section, I study the placement of the public housing projects in my sample.
Understanding these dynamics is essential for several reasons. 
First, the site-selection process for mid-century public housing was a significant criticism of the program. 
Historical case studies in several cities suggest that projects were often targeted to poorer and minority neighborhoods, both because of the program's slum clearance goals and because of local backlash in white neighborhoods against construction of projects (\cite{meyersonPoliticsPlanningPublic1955,baumanPublicHousingDreadful1994,sugrueOriginsUrbanCrisis2005}).
Second, these site selection dynamics have had important legal implications: Lawsuits in several cities have alleged that public housing site selection was discriminatory and in violation of Civil Rights Law, most famously Gautreaux v. Chicago Housing Authority (1966), but also in cities like Dallas (Walker v. HUD 1985) and Baltimore (Thompson v. HUD 2005).
Still, there has been little systematic evidence on the nationwide patterns of public housing site selection.
Finally, understanding these site-selection dynamics is important for estimating the neighborhood effects of public housing and interpreting those effects.

To begin, in Section \ref{sec:site_selection_where}, I estimate which pre-existing neighborhood characteristics predict the eventual locations of public housing projects.
I also test whether the placement of public housing projects was related to the locations of other mid-century urban policies, particularly urban renewal and the interstate highway system.
Then, in Section \ref{sec:philadelphia_site_selection}, I zoom in on a particular case study of site selection by examining the locations of proposed-but-not-built public housing in Philadelphia in Section \ref{sec:philadelphia_site_selection}.
This example illustrates the political dynamics around site selection decisions.

Finally, in Section \ref{sec:site_selection_who}, I examine whether the racial composition of the projects themselves varied systematically with the initial characteristics of the neighborhoods in which they were built.

\subsection{Where were the projects built?}\label{sec:site_selection_where}

Table \ref{tab:treated_vs_untreated_1940} shows the raw comparison of 1940 baseline characteristics between census tracts that eventually received a public housing project and those that did not. Treated neighborhoods had substantially higher Black population shares, lower median incomes and rents, higher unemployment rates, a higher share of homes needing major repairs, and were located closer to a central business district.

To formally test which neighborhood characteristics predict public housing placement, I estimate linear probability models relating pre-treatment tract characteristics to the likelihood of ever receiving a public housing project.
Specifically, I estimate:

\begin{equation}
    \text{Treated}_{ic} = \gamma_{c} + \boldsymbol{X}_{i,1940}'\boldsymbol{\beta} + \boldsymbol{Z}_{i}'\boldsymbol{\delta} + \epsilon_{ic}
\end{equation}

where $\text{Treated}_{ic}$ is an indicator for whether tract $i$ in county $c$ received any public housing project between 1941 and 1973. The vector $\boldsymbol{X}_{i,1940}$ contains 1940 neighborhood characteristics, and $\boldsymbol{Z}_{i}$ includes distance to the nearest interstate highway and an indicator for whether a tract was treated by urban renewal. I include county fixed effects $\gamma_c$ to absorb systematic differences in placement across local public housing authorities.

Columns 1-3 of Table \ref{tab:site_selection_1940_lpm} estimate the model without the $\boldsymbol{Z}_{i}$ controls, while Column 4 adds them to examine whether the relationship between 1940 characteristics and public housing placement is robust to controlling for subsequent federal interventions.
I adjust standard errors for spatial correlation following \textcite{conleyGMMEstimationCross1999}, allowing for correlation of residuals across census tracts within a 2-kilometer radius.

I chose variables based on historical narratives of public housing site selection and the program's stated intentions.
First, as described in Section \ref{sec:background}, the public housing program was largely intended as a "slum clearance" program, so we might expect that neighborhoods with lower socioeconomic status and worse housing market conditions would be more likely to receive public housing.
Second, historical case studies in multiple cities have documented the key role of race in determining where public housing was built (e.g. \cite{hirschMakingSecondGhetto1998, baumanPublicHousingDreadful1994}), so I include the Black population share as a key predictor of public housing placement.
I also include the HOLC redlining designation to capture long-term patterns of racial segregation, discrimination, and disinvestment. 

The results from these linear probability models are shown in Columns 1-4 in Table \ref{tab:site_selection_1940_lpm}. 
Column 1 shows a parsimonious model with several key neighborhood characteristics, while Column 2 shows a more saturated model including all neighborhood characteristics.
Column 3 adds the share of housing units deemed in need of major repairs in 1940, which is missing for about 5\% of tracts.
Consistent with the historical narrative, I find that public housing projects tended to be targeted towards poorer, minority neighborhoods:
Census tracts that were initially more populated, had higher Black population shares, lower median incomes and labor force participation rates, and had higher unemployment rates were more likely to receive public housing projects during this period.
Public housing was also more likely to be built in neighborhoods that redlined, had lower rents, and had higher share of homes needing repairs, although these estimates do not reach statistical significance in the fully saturated models.
These findings reflect the slum clearance motivation of the public housing program and are also consistent with historical accounts emphasizing the racial targeting of public housing siting:
Even after controlling for local economic and housing market conditions, the Black population share remains a strong and significant predictor of public housing placement.

Column 4 additionally tests whether neighborhoods that ultimately received public housing were more likely to be affected by two other transformative mid-century urban policies: urban renewal and the interstate highway system.
Evidence and historical narrative suggest that these programs may have affected many of the neighborhoods that were also targeted for public housing.
The Urban Renewal program, in particular, was directly related to public housing in many cities, as public housing projects were used to house individuals displaced by urban renewal (\cite{baumanPublicHousingDreadful1994, hirschMakingSecondGhetto1998}).
Neighborhoods that received public housing were also much more likely to be affected by urban renewal, whereas proximity to an interstate highway was not significantly related to public housing placement.
This result further confirms the interplay between these two programs and motivates accounting for urban renewal in my empirical strategy in Section \ref{sec:neighborhood_effects}.\footnote{These results are consistent with similar exercises in \textcite{harrisFirstEraAmerican2025} and \textcite{masseyPublicHousingConcentration1993}.}

Quantitatively, these effects are sizable. 
The baseline probability that a census tract in the sample received a public housing project between 1941 and 1973 is 12.6\%.
%Based on the estimates in Column (4), the coefficient on the redlined indicator suggests that neighborhoods designated as redlined were 4.3 percentage points more likely to receive public housing in subsequent decades, representing a 33.8\% increase relative to the baseline probability.
Based on the estimates in Column (4), a one standard deviation increase in Black population share increased public housing selection probability by 3.3 percentage points (26.4\% increase), a one standard deviation increase in unemployment rate increased the probability by 4.4 percentage points (34.7\% increase), and a one standard deviation decrease in median income increased the probability by 2.1 percentage points (16.5\% increase). 
Finally, neighborhoods that were eventually targeted by urban renewal were 8.8 percentage points (70\%) more likely to receive public housing.


\subsection{A Philadelphia Case Study}\label{sec:philadelphia_site_selection}
Historical accounts of public housing site selection have highlighted the political battles that surrounded the placement of projects in particular cities, for example, in Chicago (\cite{meyersonPoliticsPlanningPublic1955}), Philadelphia (\cite{baumanPublicHousingDreadful1994}), and Detroit (\cite{sugrueOriginsUrbanCrisis2005}).
These accounts have emphasized the conflict between public housing authorities and white working-class neighborhoods that resisted the construction of public housing projects in their communities.
Here, I present a case study of site selection in Philadelphia, drawing on the historical treatment and maps from \textcite{baumanPublicHousingRace1987}. 
This case study is illustrative of the political dynamics that shaped site selection decisions in many cities (\cite{huntWas1937US2005}).

Following the 1954 Housing Act, the Philadelphia Housing Authority proposed 21 public housing projects across the city. 
Upon announcement, many of these proposed projects faced significant backlash from local white communities, leading to their eventual cancellation.
I identified these proposed Philadelphia public housing locations by scanning and georeferencing a historical map from \textcite{baumanPublicHousingRace1987}, shown in Figure \ref{fig:bauman_philadelphia_fan}.
I georeferenced the historical map in QGIS, aligning it with a modern basemap using identifiable landmarks, such as major roads and rivers.
Then, I located the proposed public housing sites on the georeferenced map and matched these points to 1950 census tracts. 
In total, I identify 12 neighborhoods with proposed-but-not-built public housing sites in Philadelphia.

I then compare the baseline characteristics of proposed-but-not-built project locations to those of actual public housing locations in Philadelphia.
Table \ref{tab:philadelphia_placebo_balance} presents the balance table, which shows that the initial characteristics of the proposed locations were very different from those of actual public housing locations.
Most notably, these proposed-but-not-built project locations had very low initial Black population shares.
They were also further from the central business district, less populated, and had lower unemployment rates.
These differences illustrate the dynamics of site selection in one particular city, where whiter neighborhoods often resisted public housing construction.
They also show that an identification strategy based on proposed-but-not-built locations would likely be invalid, as these locations likely do not represent a good counterfactual for actual public housing locations.



\subsection{Did project demographics vary with neighborhood characteristics?}\label{sec:site_selection_who}

I now examine how the racial composition of the public housing projects varied with the initial neighborhood characteristics in which they were built. 
The key question is whether projects reinforced existing patterns of residential segregation or disrupted them.
I focus on the share of Black residents in each project, using data from treated census tracts in my balanced sample for which project-level racial composition is available from the 1970s.
Ideally, I would observe the racial composition of the projects at the time of construction, but I do not have this information.
% TODO: Review this
Still, to the extent that the racial composition of the projects was relatively stable over time, the 1970s data should provide a reasonable proxy for the initial demographics of the projects.

Formally, I regress the project Black share in each tract $i$ in county $c$ on the neighborhood characteristics measured in the decade before construction:
\begin{equation}
    \text{Project Black Share}_{ic} = \gamma_c + \boldsymbol{X}_{i,t-1}'\boldsymbol{\beta} + \epsilon_{ic}
\end{equation}

The unit of analysis is the tract, with one observation per treated tract.
For the 75\% of treated tracts containing a single public housing project, Project Black Share$_{ic}$ is simply the Black share in that project.
For tracts with multiple projects, I define $t$ as the construction decade of the first project to open in the tract, and Project Black Share${ic}$ is the population-weighted average Black share across all projects that opened in that first decade.
Whereas the regressions in Section \ref{sec:site_selection_where} used 1940 covariates (or later urban policy exposure) for all projects, here I match each tract to neighborhood characteristics from the decade immediately preceding its construction to better capture local conditions at the time of development.
The results are shown in Table \ref{tab:project_targeting}.
Column (1) presents a parsimonious specification with only baseline Black share and median income, column (2) includes a fuller set of neighborhood characteristics, including rent, population, unemployment, distance to the CBD, and redlining status, while column (3) adds county fixed effects.
I again include the distance from an interstate highway and urban renewal designation.

The core result is robust across specifications: public housing projects largely matched the racial composition of their surrounding neighborhoods, reinforcing rather than disrupting existing residential segregation patterns.
Based on the coefficient in column (3), a 10 percentage point increase in baseline neighborhood Black share predicts a 2.7 percentage point increase in project Black share.
Furthermore, projects built in poorer neighborhoods, even conditional on race, tended to be more heavily Black, as indicated by the negative coefficient on median income and positive coefficient on unemployment rate. 

