\section{Potential Effects of Public Housing}\label{sec:framework}
In this section, I outline the main channels through which the construction of public housing projects might affect neighborhood outcomes.

First, public housing construction may mechanically alter the racial and socioeconomic composition of neighborhoods.
Tenant selection policies, income limits, and subsidized rents determined who occupied these projects, thereby altering the demographic makeup of the areas where they were built.

Second, these compositional changes could trigger endogenous sorting by private households.
If individuals have preferences for their neighbors' race or income, the arrival of public housing residents might prompt some existing residents to relocate.
Historical work, such as \textcite{jacksonCrabgrassFrontierSuburbanization1985}, argues that part of mid-century ``white flight'' from central cities was a response to the public housing program.
However, empirical evidence on this point remains limited. 

Third, the physical structure of public housing projects could create built-environment externalities.
Replacing substandard or vacant structures with new housing could improve neighborhood quality and increase neighborhood desirability, as shown for private housing upgrades by \textcite{rossi-hansbergHousingExternalities2010} and for subsidized housing by \textcite{ellenDoesFederallySubsidized2007}. 
Conversely, large superblocks or architecturally incongruous high-rise towers might be viewed as local disamenities, and indeed, criticism of the mid-century public housing program often focused on the design of the projects \parencite{jacobsDeathLifeGreat1961,newmanCreatingDefensibleSpace1997}.

Fourth, public housing construction may generate broader social and market externalities.
A frequent concern is that concentrated poverty within large projects generated social disorder and elevated crime, with potential spillovers into surrounding neighborhoods \parencite{sandlerExternalitiesPublicHousing2017}.
Conversely, visible public investment could signal neighborhood revitalization and spur private reinvestment, improving local conditions \parencite{rossi-hansbergHousingExternalities2010}.
The direction of these effects may vary across contexts. Modern evidence on the LIHTC program suggests that subsidized housing investments may be a positive amenity in distressed areas but may be perceived negatively in more affluent ones \parencite{diamondWhoWantsAffordable2019}.

I explore these mechanisms empirically in Section \ref{sec:heterogeneity}.