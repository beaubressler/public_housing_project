\section{Introduction}

Public housing represents one of the most ambitious yet controversial urban policy initiatives in 20th century United States. Between the 1930s and 1970s, the United States government constructed approximately 1.4 million housing units with the explicit goals of neighborhood improvement, addressing urban housing shortages, and providing affordable housing (\cite{schwartzHousingPolicyUnited2021}). 

The legacy of this massive intervention in America's urban landscape remains deeply contested. Critics have attributed the program with creating and entrenching racial and economic segregation while accelerating urban decline (\cite{rothsteinColorLawForgotten2018, hirschMakingSecondGhetto1998, masseyAmericanApartheidSegregation2003}). Others scholars contend that public housing has been unfairly maligned by its worst examples, and that its poor reputation partially reflects broader social and economic challenges faced by American cities in the mid-century (\cite{baumanPublicHousingDreadful1994, bloomPublicHousingMyths2015}). These competing narratives raise fundamental questions about the relationship between government intervention, urban development, and racial inequality, questions that remain relevant as cities continue to grapple with affordable housing crises today.

Despite extensive commentary about the significance of the US public housing program and its legacy, a dearth of data has limited the ability to thoroughly evaluate these claims on a national scale. Most studies focus on single or few housing authorities, examine only short-term or contemporaneous outcomes, or employ aggregate data at the city or county level. 

This paper addresses this gap by constructing the most comprehensive dataset to date on American public housing. Combining previously digitized data, public data sources, and newly digitized material, I document the locations, construction dates, and characteristics of public housing projects containing over 1 million units nationwide. This dataset enables me to systematically analyze both the determinants of project locations and their subsequent impacts on the evolution of neighborhoods.

Using this new public housing dataset with 70 years of data on Census tract-level demographic and socioeconomic characteristics, I address two fundamental questions: First, what neighborhood characteristics determined where public housing was constructed? Using probit and linear probability models, I find that public housing was significantly more likely to be built in previously redlined neighborhoods with higher Black population shares, lower median incomes, and higher unemployment rates. These estimates are consistent with both "slum clearance" objectives and of historical accounts of discriminatory site selection (\cite{hirschMakingSecondGhetto1998}).

Second, what effects did public housing construction have on American neighborhoods in both the short- and long-term? Employing a matched difference-in-differences strategy, I compare each neighborhood that received a public housing project to a neighborhood that was on a similar trajectory before the project was built. I conduct a similar exercise for the neighborhoods surrounding public housing neighborhoods to estimate the geographic spillovers of the projects. 

In the public housing ("treated") neighborhoods, I find sizable declines in median rents and large increases in the total population and number of housing units. These population increases are concentrated among the Black population. Finally, I find large declines in median incomes and various other measures of socioeconomic status, including lower in high school graduation rates and labor force participation rates, and increases in unemployment rates.

These effects also spilled over beyond the immediate project neighborhoods. Nearby neighborhoods experienced long-term socioeconomic decline, rent decreases, increases in Black population share, and possible reductions in home values and housing units. This evidence suggests that public housing projects had a "degentrifying" effect that radiated outward into surrounding communities. 

This paper contributes to several literatures in urban and public economics and economic history. First, it extends work on affordable housing interventions (\cite{baum-snowEffectsLowIncome2009, diamondWhoWantsAffordable2019}) and place-based policies (\cite{collinsSlumClearanceUrban2013, lavoiceLongRunImplicationsSlum, rossi-hansbergHousingExternalities2010}) by providing estimates of the neighborhood effects of one of the largest urban policies in American history. More directly, a recent literature has studied the effects of the HOPE VI public housing demolitions in recent decades 
on neighborhood (\cite{tachPublicHousingRedevelopment2017, blancoKnockingItMixing2023, aliprantisBlowingItKnocking2015, sandlerExternalitiesPublicHousing2017, almagroUrbanRenewalInequality2023}) and individual (\cite{haltiwangerChildrenHOPEVI2024, chynMovedOpportunityLongRun2018}) outcomes. However, there has been little work studying effects of the construction of the projects.

My findings complement earlier work on public housing and the concentration of poverty rates in central cities, using data compiled from one or a small number of cities (\cite{carterPolarisationPublicHousing1998, masseyPublicHousingConcentration1993a}).
\textcite{shesterLocalEconomicEffects2013, shesterConcreteMeasuresRise2019} uses newly digitized data from HUD (discussed in Section \ref{sec:data}) to study the effect of public housing construction on a variety of county and city-level outcomes. The most related contribution is that of \textcite{guennewig-moenertPublicHousingDesign2024}, who studies the effects of the construction of public housing projects by the New York City Housing Authority (NYCHA) on the population, racial composition, and welfare of New York City neighborhoods. I focus on a wider set of cities, allowing me to characterize the effects of public housing in the United States more broadly\footnote{Indeed, scholars have argued that the NYCHA is in many ways an exceptional housing authority in terms of its effectiveness in dealing with many of the factors that have plagued public housing authorities in the U.S. (\cite{bloomPublicHousingThat2008}). In fact, NYCHA was a relatively small participant in HOPE VI, and did not demolish any of its high-rise projects through the program (\cite{schwartzHousingPolicyUnited2021})}, and use a different empirical approach that I argue is more suited to my setting. Contemporaneous work by Harris (2025) builds a dataset of early public housing projects, focusing on the role of these on reducing the concentration of poverty when they were first built. My paper looks at longer-run outcomes, studies a broader set of  outcomes, and focuses on a later period of the public housing program, speaking more to its eventual decline.
%\footnote{As discussed in the Appendix, the stacked spatial difference-in-differences strategy used by \textcite{guennewig-moenertPublicHousingDesign2023} does not yield good counterfactuals for many of my cities, whose Census tracts are often larger and further apart than those of New York City.}.

This paper also contributes to the economic literature on the emergence and consequences of segregation in the United States. While prior research has documented migration responses to neighborhood demographic change (\cite{shertzerRacialSortingEmergence2019a, boustanWasPostwarSuburbanization2010}), my analysis quantifies how federal housing policy directly influenced neighborhood sorting, and demonstrates the importance of urban policy in shaping segregation in American neighborhoods.


The paper proceeds as follows. Section \ref{sec:background} describes the history of the public housing program. Section \ref{sec:data} describes data sources and construction. Section \ref{sec:site_selection} analyzes the determinants of public housing site selection. Section \ref{sec:neighborhood_effeffects} presents the empirical strategy and estimates of neighborhood effects. Section \ref{sec:conclusion} concludes.

