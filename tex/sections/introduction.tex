\section{Introduction}

\begin{displayquote}\small
The Congress hereby declares that the general welfare and security of the Nation and the health and living standards of its people require housing production and community development sufficient to remedy the housing shortage, eliminate substandard housing through the clearance of slums and blighted areas, and achieve as soon as feasible the goal of a decent home and suitable living environment for every American family.
\end{displayquote}
\hfill \textit{--- Housing Act of 1949}


%Public housing represents one of the most ambitious yet controversial urban policy initiatives of the 20th century in the United States.
%Providing affordable and quality housing to low-income families has long been a central challenge for American cities.
%One of the most significant federal responses to this challenge was the mid-century public housing program. 
Between the 1930s and 1970s, the United States constructed approximately 1.4 million federally-funded public housing units in one of the most ambitious urban policy initiatives of the 20th century (\cite{schwartzHousingPolicyUnited2021}).
Designed to clear slums, address urban housing shortages, and provide affordable housing to low-income families, the American public housing program has, to many, become synonymous with policy failure.
Prominent accounts in urban history and sociology have argued that the program created and entrenched racial and economic segregation (e.g., \cite{rothsteinColorLawForgotten2017, hirschMakingSecondGhetto1998, masseyAmericanApartheidSegregation1993}), and accelerated mid-century urban decline (\cite{jacksonCrabgrassFrontierSuburbanization1985}).
Large, megablock projects such as Cabrini-Green, Pruitt-Igoe, and the Robert Taylor Homes became infamous, criticized for concentrating poverty, promoting crime, and destroying the urban fabric of neighborhoods (\cite{jacobsDeathLifeGreat1961,newmanDefensibleSpaceCrime1972}).
Federal policy since the 1970s has reflected this negative view through a retreat from public housing:
The government has shifted funding toward market-based alternatives like Housing Choice Vouchers and the Low Income Housing Tax Credit, while simultaneously dismantling the existing stock through the HOPE VI program, which has funded the demolition or removal of roughly 30\% of the original federal public housing stock since the early 1990s (\cite{schwartzHousingPolicyUnited2021}).
%The Faircloth Amendment of 1998 capped the total number of units at 1998 levels, limiting the ability of public housing authorities to build new units.

Yet the evidence for public housing's failure is mixed.
Some scholars contend that the program has been unfairly maligned by its most notorious examples and argue its poor reputation partially reflects broader social and economic challenges faced by American cities in the post-war period (\cite{baumanPublicHousingDreadful1994, bloomPublicHousingMyths2015,goetzNewDealRuins2013}).
Indeed, despite widespread criticism of the program, approximately two-thirds of public housing residents reported being "satisfied" or "very satisfied" with their housing as late as 1999 (\cite{schwartzHousingPolicyUnited2021}). Some causal evidence from economists suggests that living in public housing either has neutral or positive effects on individual outcomes (\cite{chaudhryMarcyMadisonSquare2024, currieArePublicHousing2000a,jacobPublicHousingHousing2004,pollakowskiChildhoodHousingAdult2022}), which could indicate that public housing may not have been as detrimental as some narratives suggest.
At the same time, disentangling the local effects of public housing from pre-existing neighborhood conditions remains challenging.
Given that projects were often sited in already-distressed areas as part of slum clearance efforts, did public housing cause neighborhood decline, or simply reflect the targeting of struggling areas?
A complete understanding of the public housing program has high stakes as cities grapple with housing affordability: The question of what role public housing can and should play in addressing urban housing challenges remains highly relevant.
%Moreover, as many American cities today grapple with housing affordability, the question of what role public housing can and should play in addressing urban housing challenges remains highly relevant.

Resolving these debates requires systematic evidence on where public housing was built and how it shaped neighborhoods over the long run.
Yet such an analysis has been hindered by the lack of a comprehensive public dataset linking projects to their construction dates and precise locations.
This paper addresses this research gap by constructing a new dataset on American public housing.
Combining previously digitized data, public data sources, and newly digitized materials, I obtain the locations, construction dates, and characteristics, such as project size and racial composition, of over 8,000 public housing projects containing over 1 million units nationwide.
I then integrate this dataset with a consistent-area panel of Census tract-level data from 1930-2010 that I build using both existing Census tract data and geolocated full count Census data from 1930 and 1940, all concorded to consistent tract boundaries.
I also merge these data with three other sources: Home Owners' Loan Corporation (HOLC) ``redlining'' maps from the late 1930s,
%\footnote{Recent research has shown that the HOLC maps themselves likely did not significantly influence lending decisions and should not be thought of as redlining maps (\cite{fishbackNewEvidenceRedlining2024}). However, these maps provide detailed information about neighborhood characteristics as of the late 1930s.}
maps with the locations of Urban Renewal projects, and the locations of interstate highways.
This new dataset enables me to examine several fundamental questions about the siting and long-run neighborhood consequences of the U.S. public housing program.

First, what neighborhood characteristics influenced public housing site selection?
This is important for understanding the political economy of public housing and for informing my identification strategy for estimating the neighborhood effects of public housing construction.
In particular, I am interested in whether public housing was explicitly targeted towards Black neighborhoods, as some historical accounts suggest (e.g., \cite{hirschMakingSecondGhetto1998, rothsteinColorLawForgotten2017}), or whether it was more broadly targeted towards poor and working-class neighborhoods.
To answer this, I estimate linear probability models to explore how pre-existing neighborhood characteristics, including demographic, socioeconomic, housing, and institutional factors, predict the placement of public housing projects.
I find that public housing projects were more likely to be built in neighborhoods that were initially poorer, more populated, and had higher shares of Black residents, consistent with the program's slum clearance goals and the racialized politics of housing policy in the mid-20th century.
I also find that neighborhoods that received public housing were more likely to be affected by urban renewal, another large mid-century urban policy that targeted predominantly Black neighborhoods (\cite{lavoiceLongrunImplicationsSlum2024}).
Finally, I find that public housing projects built in poorer neighborhoods and neighborhoods with higher Black population shares tended to have higher shares of Black residents themselves, indicating that public housing reinforced rather than disrupted existing residential segregation patterns.
I also show that these patterns hold in a specific instance of site selection in Philadelphia by digitizing a map of proposed-but-not-selected public housing sites from \textcite{baumanPublicHousingRace1987}:
In particular, neighborhoods proposed but rejected for public housing were initially much whiter than those that ultimately received projects.
These results are consistent with historical narratives and case study evidence about the politics of public housing site selection (e.g., \cite{meyersonPoliticsPlanningPublic1955,baumanPublicHousingDreadful1994,hirschMakingSecondGhetto1998}).
% 9/25/2025: Possible sentence: Still, neighborhoods that received public housing were not necessarily extremely Black when the projects were built.

Second, what were the short- and long-run effects of public housing construction on both the recipient and surrounding neighborhoods? 
To answer this question, I employ a stacked matched difference-in-differences strategy that explicitly compares each public housing neighborhood to a matched never-treated control neighborhood based on pre-treatment characteristics. 
For each neighborhood that received a public housing project, I use nearest-neighbor propensity score matching to identify a comparable control neighborhood in the same county based on the pre-treatment characteristics that I demonstrated were key determinants of site selection.
I then treat each matched pair as a separate "sub-experiment" and stack these sub-experiments into a single analytic dataset. 
The specification includes matched-pair-by-year fixed effects, ensuring that each treated neighborhood is compared only to its matched control in each year, while controlling for time-varying shocks that affect both neighborhoods in the same pair.
Identification stems from the fact that federal funding for public housing was limited, and local housing authorities could not build projects in every eligible neighborhood, creating variation in project placement across otherwise similar areas.
%For example, if a particular treated neighborhood in Chicago was a poor, majority-Black neighborhood in 1950, it would be matched to a similar poor, majority-Black neighborhood in Chicago that did not receive public housing, and the fixed effect would control for any shocks affecting poor Black neighborhoods in Chicago in each year. 
%This design ensures each treated neighborhood is compared only to its matched control both before and after public housing construction, while controlling for a variety of potential confounding factors and allowing me to test for differential pre-trends.
Importantly, this stacked framework also avoids the econometric pitfalls that arise in staggered adoption settings when treatment effects are heterogeneous across cohorts and over time (\cite{wingStackedDifferenceDifferences2024}).
I extend this methodology to analyze the geographic spillovers of the projects by conducting a similar exercise for neighborhoods adjacent to public housing neighborhoods, thereby estimating the broader geographic effects of the projects. 

In the public housing ("treated") neighborhoods, I find large increases in total population driven particularly by increases in the Black population, resulting in substantial increases in Black population shares.
I also find large decreases in median rents, along with declines in median incomes and various other measures of economic well-being, including lower labor force participation rates and higher unemployment rates.
These results, in conjunction with the site-selection results, suggest that while public housing was targeted toward poorer, minority neighborhoods, the construction of the projects further accelerated demographic changes and socioeconomic decline in these neighborhoods over the long run.

In terms of geographic spillovers, I find that adjacent neighborhoods experienced decreases in median incomes but showed little evidence of significant changes in racial composition, population, or rents on average.
This suggests that while public housing may have had some negative spillovers on surrounding neighborhoods, these effects were more muted than some historical narratives have suggested (e.g. \cite{jacksonCrabgrassFrontierSuburbanization1985}).

I also explore the heterogeneity of these effects along several dimensions.
First, I explore heterogeneity based on the neighborhood's initial racial composition.
I find that treated and nearby neighborhoods that initially had Black population shares within a potential "tipping range" (between 1 and 12\%, based on work by \cite{cardTippingDynamicsSegregation2008}) saw outflows of white residents. In contrast, neighborhoods that were initially more Black did not.
This suggests that public housing construction triggered racial tipping dynamics in some neighborhoods.
I also find that the long-run effects of public housing construction on neighborhoods were larger for projects built before 1960, and that the impact of the projects I detect was driven mainly by neighborhoods not treated by the urban renewal program.

Finally, I explore the implications of these neighborhood effects for economic opportunity.
To do so, I link my dataset to data on the upward mobility of children born to low-income families born from 1978-1983 from Opportunity Atlas (\cite{chettyOpportunityAtlasMapping2018}).
I find that low-income children growing up in neighborhoods that received public housing projects experienced significantly lower upward mobility than children in matched control neighborhoods, even after controlling for neighborhood characteristics.
I also find small adverse effects on upward mobility in nearby neighborhoods, but these effects are fully accounted for by controlling for neighborhood income and demographics in 1980, with no additional effects from proximity to public housing itself.
This pattern suggests that these modest spillovers occurred through earlier neighborhood changes, rather than through persistent or independent effects of public housing on later mobility outcomes in surrounding neighborhoods.

%I also conduct two supplementary exercises using data collected for particular cities. 
%In Appendix \ref{app:chicago_land_values}, I use a long-run panel of historical land values in Chicago at small geographic scale to study the effects of public housing construction on land values.
%Using a stacked, spatial difference-in-differences strategy, I find little evidence that public housing projects had large negative effects on land values, and if anything, that small projects had positive effects on land values.

%Insert bit about heterogeneity here

This paper contributes to several literatures in urban and public economics and economic history.
First, it contributes to the literature on the local effects of affordable housing programs (\cite{baum-snowEffectsLowIncome2009, diamondWhoWantsAffordable2019}) and revitalization policies (\cite{collinsSlumClearanceUrban2013, lavoiceLongrunImplicationsSlum2024, rossi-hansbergHousingExternalities2010}) by estimating the neighborhood effects of one of the largest urban policies in American history.
More directly, recent literature has studied the effects of HOPE VI public housing demolitions on neighborhood (\cite{tachPublicHousingRedevelopment2017, blancoKnockingItMixing2025, aliprantisBlowingItKnocking2015, sandlerExternalitiesPublicHousing2017, almagroUrbanRenewalInequality2023}) and individual (\cite{haltiwangerChildrenHOPEVI2024, chynMovedOpportunityLongRun2018}) outcomes.
This literature has focused on the demolitions of the most distressed public housing projects, and much of it has been focused on a single city (Chicago). 
There are several reasons to think these estimates of the effects of demolitions may not generalize to those of public housing construction.
First, the projects demolished through the HOPE VI program are a selected set of the most distressed projects. They are thus not necessarily representative of the public housing program as a whole.
Second, the effects of project construction likely depended on existing neighborhood conditions, which likely varied in ways that they did not for demolitions.
New projects may have served as local amenities in some neighborhoods but disamenities in others.
Finally, the effects in Chicago might not necessarily generalize to other cities. 

My findings also complement earlier work from other social sciences on public housing and the concentration of poverty in central cities, which generally used data compiled from one or a small number of cities (e.g. \cite{carterPolarisationPublicHousing1998, masseyPublicHousingConcentration1993}).
Relative to this older literature, I use a dataset covering a much broader set of cities, projects, and time periods, and apply modern econometric techniques to estimate the causal effects of public housing construction on neighborhoods.

My paper adds to the small literature estimating the effects of public housing construction.
\textcite{shesterLocalEconomicEffects2013} uses digitized data from HUD that contains all federally funded public housing projects built until 1973, with locations at only the locality level (discussed in Section \ref{sec:data}) to study the effect of public housing construction on a variety of county- and city-level outcomes. 
This paper finds that cities in the same state with more public housing construction experienced decreases in median property values, median family income, and population density.
\textcite{shesterConcreteMeasuresRise2019} uses the same dataset to study the effects of public housing construction on the rise in single motherhood at the MSA level.
I build on this work by merging precise locations to these data, allowing me to study neighborhood effects.

The most closely related contribution is that of \textcite{guennewig-moenertPublicHousingPreferences2025}, who studies the effects of public housing construction in New York City, specifically on neighborhood racial composition and rents, along with welfare estimates based on a structural model.
I compile data on public housing projects across the country, enabling me to characterize the effects of public housing in the United States more broadly and to speak to broader historical debates about the program.
Scholars have argued that the New York City Housing Authority (NYCHA) was in many ways exceptional in terms of its effectiveness in dealing with many of the factors that have plagued public housing authorities in the U.S. \parencite{bloomPublicHousingThat2008}.\footnote{For example, NYCHA was a relatively small participant in HOPE VI and did not demolish any of its high-rise projects through the program (\cite{schwartzHousingPolicyUnited2021}).}
Our empirical approaches also differ: he uses variation in proximity to public housing to define the control group, while I use similarity in pre-treatment characteristics.
%In Appendix X, I show that a pure distance-based approach to define the control group yields significant pre-trend violations in my setting.
Another related contribution is contemporaneous work by \textcite{harrisFirstEraAmerican2025}, who studies the short-run neighborhood effects in the first two decades of the program and finds that public housing initially reduced poverty concentration in recipient neighborhoods.

% TODO: Could write more here
Finally, my paper contributes to the economic literature on the emergence and consequences of segregation in the United States.
While prior research has documented migration responses to neighborhood demographic change (\cite{shertzerRacialSortingEmergence2019a, boustanWasPostwarSuburbanization2010}), my paper addresses one way in which federal housing policy directly influenced neighborhood sorting. 
More broadly, this paper contributes to the literature and debates around what role public policy has played in shaping racial and economic segregation in American cities (\cite{rothsteinColorLawForgotten2017,trounstineSegregationDesignLocal2018,boustanRacialResidentialSegregation2013a, loganRacialResidentialSegregation2025}).


The paper proceeds as follows.
Section \ref{sec:background} describes the history of the public housing program.
Section \ref{sec:framework} outlines the potential channels through which public housing may affect neighborhoods.
Section \ref{sec:data} describes data sources and construction.
Section \ref{sec:site_selection} analyzes the determinants of public housing site selection.
Section \ref{sec:neighborhood_effects} presents the empirical strategy and estimates of neighborhood effects.
Section \ref{sec:heterogeneity} explores heterogeneity in these effects and examines potential mechanisms.
Section \ref{sec:opportunity_insights} links public housing to long-run economic opportunity using data from the Opportunity Atlas.
Section \ref{sec:conclusion} concludes.

