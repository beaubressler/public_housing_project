\section{Introduction}

\begin{displayquote}\small
The Congress hereby declares that the general welfare and security of the Nation and the health and living standards of its people require housing production and community development sufficient to remedy the housing shortage, eliminate substandard housing through the clearance of slums and blighted areas, and achieve as soon as feasible the goal of a decent home and suitable living environment for every American family.
\end{displayquote}
\hfill \textit{--- Housing Act of 1949}



Public housing represents one of the most ambitious yet controversial urban policy initiatives in 20th century United States.
Between the 1930s and 1970s, the United States government constructed approximately 1.4 million housing units with the explicit goals of neighborhood improvement, addressing urban housing shortages, and providing affordable housing (\cite{schwartzHousingPolicyUnited2021}). 

%The legacy of this large-scale intervention in America's urban landscape remains deeply contested.
The broader historical narrative has argued that the program was largely a failure. Prominent historical narrratives have attributed the public housing program with creating and entrenching racial and economic segregation (e.g. \cite{rothsteinColorLawForgotten2018, hirschMakingSecondGhetto1998, masseyAmericanApartheidSegregation2003})) and accelerating urban decline (\cite{jacksonCrabgrassFrontierSuburbanization2006}). Large, megablock projects were criticized for concentrating poverty, promoting crime, and destroying the urban fabric of neighborhoods (\cite{jacobsDeathLifeGreat1992,newmanCreatingDefensibleSpace1997})
The history of the policy reflects this negative view, as the federal government has shifted away from the public housing program, largely moving towards housing vouchers and the Low Income Housing Tax Credit as the primary means of providing affordable housing. Through the HOPE VI program, public housing authorities have demolished about 30\% of the total public housing stock through the HOPE VI program (\cite{schwartzHousingPolicyUnited2021}), while the Faircloth Amendment of 1998 capped the total number of public housing units that could be operated by public housing authorities at 1998 levels.

Still, others scholars contend that the mid-century public housing program has been unfairly maligned by its most notorious examples, and that its poor reputation partially reflects broader social and economic challenges faced by American cities in the mid-century (\cite{baumanPublicHousingDreadful1994, bloomPublicHousingMyths2015}). And despite widespread criticism of the program, in 1999, about two-thirds of public housing residents reported being "satisfied" or "very satisfied" with their housing (\cite{schwartzHousingPolicyUnited2021}).
And as many American cities today grapple with housing affordability, the question of what role public housing can and should play in addressing urban housing challenges remains highly relevant.



Despite extensive commentary about the significance of the US public housing program and its transformative legacy on American cities, a lack of data has limited the ability to thoroughly evaluate the long-term effects of the program on American neighborhoods.
This paper addresses this gap by constructing a new dataset on American public housing. Combining previously digitized data, public data sources, and newly digitized material, I obtain the locations, construction dates, and characteristics of public housing projects containing over 1 million units nationwide.
I combine this dataset with eight decades of neighborhood-level Census data to study several fundamental questions about the public housing program.
First, what neighborhood characteristics determined where public housing was constructed, and did the chracteristics of the neighborhoods determine the chracteristics of the projects themselves? 
Second, what were the short- and long-run effects of the construction of the projects on the neighborhoods that received them, as well as on surrounding neighborhoods? 
Employing a matched difference-in-differences strategy, I compare each neighborhood that received a public housing project to a neighborhood that was on a similar trajectory before the project was built. I conduct a similar exercise for the neighborhoods surrounding public housing neighborhoods to estimate the geographic spillovers of the projects. 

In the public housing ("treated") neighborhoods, I find sizable declines in median rents and large increases in the total population and number of housing units. These population increases are concentrated among the Black population. Finally, I find large declines in median incomes and various other measures of socioeconomic status, including lower in high school graduation rates and labor force participation rates, and increases in unemployment rates.

These effects also spilled over beyond the immediate project neighborhoods. Nearby neighborhoods experienced long-term socioeconomic decline, rent decreases, increases in Black population share, and possible reductions in home values and housing units. This evidence suggests that public housing projects had a "degentrifying" effect that radiated outward into surrounding communities. 

This paper contributes to several literatures in urban and public economics and economic history. First, it extends work on the local effects of affordable housing programs (e.g. \cite{baum-snowEffectsLowIncome2009, diamondWhoWantsAffordable2019}) and revitalization policies (\cite{collinsSlumClearanceUrban2013, lavoiceLongRunImplicationsSlum, rossi-hansbergHousingExternalities2010}) by providing estimates of the neighborhood effects of one of the largest urban policies in American history, which historical narratives suggest may have had large and transformative effects on American neighborhoods.
More directly, a recent literature has studied the effects of the HOPE VI public housing demolitions in recent decades 
on neighborhood (\cite{tachPublicHousingRedevelopment2017, blancoKnockingItMixing2023, aliprantisBlowingItKnocking2015, sandlerExternalitiesPublicHousing2017, almagroUrbanRenewalInequality2023}) and individual (\cite{haltiwangerChildrenHOPEVI2024, chynMovedOpportunityLongRun2018}) outcomes. However, there has been little work studying effects of the construction of the projects.

My findings complement earlier work on public housing and the concentration of poverty rates in central cities, using data compiled from one or a small number of cities (\cite{carterPolarisationPublicHousing1998, masseyPublicHousingConcentration1993a}).
\textcite{shesterLocalEconomicEffects2013, shesterConcreteMeasuresRise2019} use digitized data from HUD (discussed in Section \ref{sec:data}) to study the effect of public housing construction on a variety of county and city-level outcomes. The most related contribution is that of \textcite{guennewig-moenertPublicHousingDesign2024}, who studies the effects of the construction of public housing projects by the New York City Housing Authority (NYCHA) on the population, racial composition, and welfare of New York City neighborhoods. I focus on a wider set of cities, allowing me to characterize the effects of public housing in the United States more broadly\footnote{Indeed, scholars have argued that the NYCHA is in many ways an exceptional housing authority in terms of its effectiveness in dealing with many of the factors that have plagued public housing authorities in the U.S. (\cite{bloomPublicHousingThat2008}). In fact, NYCHA was a relatively small participant in HOPE VI, and did not demolish any of its high-rise projects through the program (\cite{schwartzHousingPolicyUnited2021})}, and use a different empirical approach that I argue is more suited to my setting. Contemporaneous work by Harris (2025) builds a dataset of early public housing projects, focusing on the role of these on reducing the concentration of poverty when they were first built. My paper looks at longer-run outcomes, studies a broader set of  outcomes, and focuses on a later period of the public housing program, speaking more to its eventual decline.
%\footnote{As discussed in the Appendix, the stacked spatial difference-in-differences strategy used by \textcite{guennewig-moenertPublicHousingDesign2023} does not yield good counterfactuals for many of my cities, whose Census tracts are often larger and further apart than those of New York City.}.

This paper also contributes to the economic literature on the emergence and consequences of segregation in the United States. While prior research has documented migration responses to neighborhood demographic change (\cite{shertzerRacialSortingEmergence2019a, boustanWasPostwarSuburbanization2010}), my analysis quantifies how federal housing policy directly influenced neighborhood sorting, and demonstrates the importance of urban policy in shaping segregation in American neighborhoods.

The paper proceeds as follows. Section \ref{sec:background} describes the history of the public housing program. Section \ref{sec:data} describes data sources and construction. Section \ref{sec:site_selection} analyzes the determinants of public housing site selection. Section \ref{sec:neighborhood_effeffects} presents the empirical strategy and estimates of neighborhood effects. Section \ref{sec:conclusion} concludes.

