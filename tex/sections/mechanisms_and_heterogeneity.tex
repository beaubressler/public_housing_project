\section{Mechanisms and Heterogeneity}\label{sec:heterogeneity}

Section \ref{sec:framework} outlined several potential channels through which public housing construction might affect neighborhoods: mechanical compositional channels from adding public housing residents, endogenous sorting responses by private households, and externalities from the build environment or concentrated poverty. In this section, I explore these mechanisms empirically.

I first present evidence that suggests that the effects of public housing construction were not purely mechanical. 


The results in Section \ref{sec:neighborhood_effects} show that public housing led to increases in Black population shares and declines in median income and rents.
In this section, I explore potential mechanisms behind theese 

My baseline results show that public housing construction increased Black population 
In this section, I explore heterogeneity in the neighborhood effects of the public housing program.



For simplicity, I show estimates at $t=2$, the third post-treatment decade, which capture the program's long-run effects.
There are several dimensions along which the effects of public housing may have varied.

First, the effects may have varied based on the initial neighborhood characteristics.
Modern evidence on the construction of affordable housing projects through the Low Income Housing Tax Credit (LIHTC) program suggests that whether the new projects are a local amenity or a disamenity depends on neighborhood characteristics (\cite{diamondWhoWantsAffordable2019}).
Furthermore, models of neighborhood tipping suggest that the initial racial composition of the neighborhood may influence the extent to which public housing construction precipitates white flight or overall racial transition (\cite{schellingDynamicModelsSegregation1971,cardTippingDynamicsSegregation2008}).
Some historical accounts of public housing argue that public housing construction may have led to neighborhood racial transition through tipping dynamics (\cite{rothsteinColorLawForgotten2017,jacksonCrabgrassFrontierSuburbanization1985}).

To test this tipping hypothesis, I examine how the effects of public housing varied across neighborhoods with different initial Black population shares.
I divide treated and nearby neighborhoods into three groups based on their baseline Black population share measured one decade before public housing construction:
Neighborhoods that are almost entirely non-Black (less than 1\% Black share), those with "medium" initial Black shares that I will consider as those in the tipping range (between 1\% and 12\%), and those with high initial Black shares (12\% or higher).
I choose 12\% for the top of the "medium" range, as it corresponds to the tipping threshold estimated by \textcite{cardTippingDynamicsSegregation2008} in 1970. 

Figure \ref{fig:baseline_black_share_heterogeneity} presents the results of this analysis at $t=2$.
I show estimates using raw population counts rather than log population, since percentage changes in Black population will mechanically be larger in neighborhoods with small initial Black populations.
I find significant differences in the effects of public housing on neighborhood racial composition, depending on the initial Black population share.
Treated neighborhoods that were almost entirely non-Black and those in the "tipping range" saw sizable increases in Black population in response to public housing construction, while treated neighborhoods in the "tipping range" also experienced significant declines in white population. In contrast, treated neighborhoods with high initial Black shares saw no change in Black population, and if anything, some increase in white population.
Similarly, nearby neighborhoods in the "tipping range" also saw sizable declines in white population, providing evidence consistent with white flight dynamics in response to public housing construction for neighborhoods in this range. 

I also test whether the effects of the projects varied depending on when they were built.
Comparing the effects of the projects built in the 1940s and 1950s to those built after 1960 may highlight how the changing political and social context surrounding public housing influenced its neighborhood effects.
Figure \ref{fig:construction_decade_heterogeneity} shows the results of this analysis at $t=2$.
I find that, particularly for population and racial composition, the effects differed substantially based on when the projects were built.
In particular, I see evidence of substantial long-run Black population increase in public housing neighborhoods, and long-run white population exit in nearby neighborhoods for projects built in the early period, but not for those built later.
I am still exploring the mechanisms behind this result.
One possibility is that this reflects the changing racial composition of American cities over this period and reflects the same racial tipping mechanisms as in Figure \ref{fig:baseline_black_share_heterogeneity}:
For earlier projects, the Black population in the average neighborhood was much lower.  

Finally, I test whether these effects differed in neighborhoods that were also affected by the urban renewal program.
This analysis both highlights potential interactions between the two programs and addresses concerns that my baseline estimates may be conflated with the effects of urban renewal.
Figure \ref{fig:urban_renewal_heterogeneity} presents the results at $t=2$, separately estimating Equation \ref{eq:matched_did} for neighborhoods that were urban renewal tracts and those that were not.
I find null effects on median income and population by race in neighborhoods that were also urban renewal tracts, suggesting that the impact of urban renewal may have outweighed any effects of public housing.
In contrast, in neighborhoods not targeted by urban renewal, I find similar effects as in the main analysis.
