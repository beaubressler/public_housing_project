\section{Conclusion}\label{sec:conclusion}

This paper provides a comprehensive analysis of the effects of the U.S. public housing program on the evolution of neighborhoods using a novel dataset on American public housing projects. By documenting the locations, construction dates, and characteristics of over 1 million public housing units nationwide, this research offers new insights into both the determinants of project placement and their subsequent effects on neighborhood composition and socioeconomic conditions.

The findings reveal clear patterns in public housing site selection that align with historical narratives. Projects were significantly more likely to be built in previously redlined neighborhoods with higher Black population shares, lower median incomes, and higher unemployment rates. Then, using a matched difference-in-differences approach, I find  substantial and persistent neighborhood transformations following public housing construction. In particular, I find that public housing construction led to large increases in the Black population share and substantial declines in socioeconomic status. These two results suggests that while the program may have been motivated by "slum clearance" objectives, site selection simultaneously reinforced and entrenched existing patterns of racial and economic segregation. 

Importantly, these effects extended beyond the immediate project neighborhoods. Nearby neighborhoods experienced similar patterns of socioeconomic decline, rent decreases, increases in Black population share, and reductions in housing value and units. This evidence suggests that public housing projects had what might be termed a "degentrifying" effect that radiated outward into surrounding communities, contributing to broader patterns of neighborhood sorting and urban decline.

These findings contribute to several literatures in urban economics and economic history while contributing to ongoing debates about the legacy of the public housing program. They extend work on place-based policies by providing estimates of neighborhood effects from one of the largest urban interventions in American history. They also add to our understanding of the economic mechanisms underlying racial segregation by quantifying how federal housing policy directly influenced neighborhood composition and sorting patterns.

%The results have implications for evaluating the welfare effects of public housing programs. While these projects addressed housing shortages and provided subsidized units to disadvantaged populations, they also generated negative externalities that affected both public housing residents and surrounding communities. These trade-offs should inform contemporary policy debates about affordable housing provision and neighborhood revitalization strategies.

In ongoing work, I explore heterogeneity in the effects of the public housing program across project characteristics, cities and time, to better understand whether the negative effects of public housing on neighborhoods can be mitigated by project characteristics. The mechanisms and channels behind the reduced-form results could also be further explored through a neighborhood choice model. I am also in the process of studying the effects of living in public housing projects on individuals' short- and long-run outcomes by linking individuals to the projects themselves.