\section{Conclusion}\label{sec:conclusion}
This paper has examined the long-run neighborhood effects of the mid-20th-century U.S. public housing program.
Using a newly constructed national dataset of more than 7,000 projects built between 1935 and 1973, linked to a consistent panel of tract-level census data spanning 1930–2010, I documented two central findings.
First, public housing projects were systematically targeted toward neighborhoods that were initially poorer, more populated, disproportionately Black, and slated for urban renewal.
These patterns confirm the program’s slum-clearance origins and the racialized politics of site selection.
Second, public housing construction had significant and persistent effects on neighborhood trajectories: treated tracts experienced long-run increases in Black population shares and sustained declines in incomes and rents, while spillover effects on nearby
neighborhoods were more limited.
Importantly, I find evidence consistent with racial tipping dynamics: neighborhoods with moderate baseline Black shares experienced substantial white population outflows following public housing construction, while neighborhoods with higher initial Black shares showed more muted responses.

Taken together, these results suggest that public housing not only reflected but also reinforced existing patterns of segregation and disinvestment, contributing quantitative evidence to longstanding debates about the role of federal policy in creating residential segregation (\cite{rothsteinColorLawForgotten2017,trounstineSegregationDesignLocal2018, loganRacialResidentialSegregation2025}).
Rather than catalyzing neighborhood improvement as envisioned in the Housing Act of 1949, the program contributed to the concentration of poverty and the persistence of racial segregation in American cities. 
The historical patterns I document have clear implications for contemporary affordable housing policy. 
Modern reforms to public housing have emphasized smaller-scale interventions, mixed-income developments, and designs that blend into existing neighborhoods rather than large, segregated projects.
My findings help explain the rationale behind these reforms: large concentrations of subsidized housing in neighborhoods undergoing racial transition accelerated segregation and demographic change.


One crucial caveat is that my results do not necessarily imply that public housing failed to benefit individual residents.
For many low-income residents, public housing may have provided better housing quality, stability, and affordability than available private-market alternatives.
In ongoing work, I am linking individuals to public housing projects using full-count Census data to study the effects on project residents and their neighbors.
This individual-level analysis will complement the neighborhood-level findings presented here by directly measuring the benefits and costs experienced by public housing residents themselves, providing a more complete assessment of the program's legacy.


%Future work will explore heterogeneity across cities and project types in greater detail, connect neighborhood-level changes to individual mobility, and compare these results to more recent housing policies such as HOPE VI and the LIHTC.


%This paper provides a comprehensive analysis of the effects of the U.S. public housing program on the evolution of neighborhoods using a novel dataset on American public housing projects. By documenting the locations, construction dates, and characteristics of over 1 million public housing units nationwide, this research offers new insights into both the determinants of project placement and their subsequent effects on neighborhood composition and socioeconomic conditions.

%The findings reveal clear patterns in public housing site selection that are mainly consistent with the mainstream historical narratives around the program.
%Projects were significantly more likely to be built in previously redlined neighborhoods with higher Black population shares, lower median incomes, and higher unemployment rates.
%Then, using a matched difference-in-differences approach, I find substantial and persistent neighborhood transformations following public housing construction.
%In particular, I find that public housing construction led to significant increases in the Black population share and substantial declines in socioeconomic status. These two results suggest that while the program may have been motivated by "slum clearance" objectives, site selection simultaneously reinforced and entrenched existing patterns of racial and economic segregation. 

%Importantly, these effects extended beyond the immediate project neighborhoods. 
%Nearby neighborhoods experienced similar patterns of socioeconomic decline, rent decreases, increases in Black population share, and reductions in housing value and units. This evidence suggests that public housing projects had what might be termed a "degentrifying" effect that radiated outward into surrounding communities, contributing to broader patterns of neighborhood sorting and urban decline.

%These findings contribute to several literatures in urban economics and economic history while contributing to ongoing debates about the legacy of the public housing program. They extend work on place-based policies by providing estimates of neighborhood effects from one of the most significant urban interventions in American history. They also add to our understanding of the economic mechanisms underlying racial segregation by quantifying how federal housing policy directly influenced neighborhood composition and sorting patterns.

%The results have implications for evaluating the welfare effects of public housing programs. While these projects addressed housing shortages and provided subsidized units to disadvantaged populations, they also generated negative externalities that affected both public housing residents and surrounding communities. These trade-offs should inform contemporary policy debates about affordable housing provision and neighborhood revitalization strategies.

%In ongoing work, I explore heterogeneity in the effects of the public housing program across project characteristics, cities, and time, to better understand whether project characteristics can mitigate the adverse effects of public housing on neighborhoods. The mechanisms and channels behind the reduced-form results could also be further explored through a neighborhood choice model. I am also studying the effects of living in public housing projects on individuals' short- and long-run outcomes by linking individuals to the projects they live in.