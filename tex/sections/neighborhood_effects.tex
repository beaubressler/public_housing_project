\section{The Effect of Public Housing Construction on Neighborhoods}\label{sec:neighborhood_effects}

Having established in the previous section that public housing projects were systematically targeted towards poorer, minority neighborhoods, I now turn to estimating the effects of the projects on neighborhood change in subsequent decades.
The central empirical challenge here is to select appropriate counterfactual neighborhoods. To do so, I employ a matched difference-in-differences approach informed by the site selection results in Section \ref{sec:site_selection_where}.
I proceed as follows. First, for each treated neighborhood, I exclude the set of census tracts that are contiguous to it from the donor pool of potential control neighborhoods, to avoid concerns about spillovers.
In Section \ref{sec:spillover_effects}, I will directly test the effects of the construction of the projects on these nearby neighborhoods themselves.
Then, I use nearest neighbor matching with replacement to select comparison neighborhoods within the same county that are similar to the treated neighborhoods before public housing is constructed, specifically in terms of characteristics that predicted public housing placement.
In particular, I match on those pre-treatment characteristics which I found in Section \ref{sec:site_selection} were strongly statistically significant predictors of the placement of public housing projects: Total population, Black population share, median income, unemployment rate, and redlined status.
I require an exact match on county and redlined status, and also whether a tract was an urban renewal tract, in order to avoid conflating the effects of these two policies.
\footnote{While the site selection analysis in Section \ref{sec:site_selection_where} presented results using log transformations for population, incomes and rents, I match on levels to ensure treated and control neighborhoods are similar in absolute terms. I use propensity score-based nearest neighbor matching with replacement, where propensity scores are estimated using logistic regression on the matching variables within strata defined by the exact matching constraints (county, redlining status, and urban renewal designation).}
In my baseline specification, I identify only the single closest matched control for each treated neighborhood. Table \ref{tab:balance_treated} shows balance statistics for the treated neighborhoods and their matched controls. 
%The matching procedure achieves good balance on all pre-treatment characteristics, with standardized differences below 0.1 for all variables.

\input{../output/balance_tables/matched_did/combined/balance_table_treated_neighborhoods.tex}


Having identified a set of control neighborhoods, I implement a stacked difference-in-differences design that compares changes in outcomes over time between each treated neighborhood and its matched controls. 
In particular, for tract $i$ in matched pair $m$ at time $t$, I estimate the following event-study specification:

\begin{equation}\label{eq:matched_did}
y_{imt} = \alpha_i + \sum_{\tau\neq-10} \beta_\tau (D_{imt}^\tau \times \text{Treated}_i) + \sum_{\tau\neq-10} (\delta_m \times D_{imt}^\tau) + \varepsilon_{imt}
\end{equation}

where $y_{imt}$ is the outcome of interest for tract $i$ in matched pair $m$ at time $t$, $\alpha_i$ are tract fixed effects, $D_{imt}^\tau = \mathbb{1}[(t - T_{im}) = \tau]$ are indicators for event time $\tau$ relative to treatment year, $\text{Treated}_i$ indicates whether tract $i$ is a treated tract, and $\delta_m$ are matched pair fixed effects. 
Each matched pair fixed effect is interacted with $D_{imt}^\tau$ to explicitly compare each tract to its matched control at each event time. The $\beta_\tau$ terms measure the difference in outcomes between treated and control neighborhoods at each time period relative to the construction of public housing.

This event-study specification allows me to examine both pre-trends and dynamic treatment effects over time.\footnote{My balancing procedure ensures that I have at least one pre-trend estimate for each matched pair. Event times t=-30 and t=30 may not be observed for all matched pairs.}
Operationally, this specification is equivalent to a stacked difference-in-differences design in which each matched pair can be thought of as its own "sub-experiment" (\cite{wingStackedDifferenceDifferences2024,blancoKnockingItMixing2023}).\footnote{Since each treated unit is matched to a single control unit, one does not need to adjust for imbalances in treatment/control composition through weighting as outlined in \textcite{wingStackedDifferenceDifferences2024}.}
By explicitly comparing matched pairs at each event time, this setup avoids econometric issues that have plagued the staggered adoption difference-in-differences setting (\cite{callawayDifferenceDifferencesStaggered2021, sunEstimatingDynamicTreatment2021}).
Standard errors are adjusted for spatial correlation following \cite{conleyGMMEstimationCross1999} within a radius of 2 kilometers.

Identification relies on the parallel trends assumption that the treated neighborhoods would have been on similar trajectories as their matched control if not for the construction of public housing. 
I argue that this is plausible for several reasons. Primarily, federal funding for public housing was limited, and local public housing authorities could not construct public housing projects in every neighborhood that met the criteria for public housing. 
There are several threats to identification that we should be cautious of. One potential concern would be if there were unobserved shocks that differentially affected the treated tracts. For example, if public housing was built in places that saw some shock in the decade preceding the completion of the public housing project, the effects of public housing would be conflated with the effects of that shock.
Furthermore, we might be concerned that neighborhoods that were chosen for public housing different on other unobservable characteristics from their matched controls.
\footnote{Other recent work on similar targeted policies has used proximity to the treatment to define the control group. For example, \textcite{blancoKnockingItMixing2023} estimate the effects of public housing regenerations in the UK on nearby neighborhoods by using slightly further neighborhoods/housing prices as a control group. In my setting, it is challenging to find a suitable control group purely using a distance-based approach that exhibits parallel pre-trend estimates. This may be due to the geographic size of Census tracts in some cities, especially in 1950. In Appendix \ref{sec:spatial_did_appendix}, I show estimates of the effects of public housing using a stacked spatial difference-in-differences approach, where the control tracts are defined purely by their proximity to the treated tracts.}


Figure \ref{fig:population_demographics_treated} presents the effects of public housing construction on population and racial composition in the treated neighborhoods. The results show large and persistent demographic changes relative to the matched control neighborhoods.
Total population in the public housing neighborhoods increased substantially, with log population rising by $\approx$ 14\% in the immediate decade following construction. This increase was driven by very large increases in the total Black population (46\%), with little effect on total white population on average. 
Relative to the control neigborhoods, Black population shares increased by 2.6 percentage points in the first decade following construction, and continued to increase up to approximately 6 percentage points in the third decade after construction (t=20). This represents a 21.7\% increase in the Black population share relative to the baseline share of 27.8\%.


Figure \ref{fig:private_population_treated} decomposes the population effects by examining changes in the estimated private population separately from the total population changes shown above.
This shows that the sizable increases in population were driven by increases in public housing population, which displaced residents in private housing.\footnote{As discussed in Section \ref{sec:data}, I do not have public housing population data for all projects, so these estimates are on a smaller sample of projects.}


  \begin{figure}[htbp]
      \centering
      \includegraphics[width=0.9\textwidth]{../output/regression_results/matched_did/combined/event_study_population_demographics_treated.pdf}
      \caption{Population and Racial Composition Effects in Treated
  Neighborhoods}
      \label{fig:population_demographics_treated}
      \footnotesize{\textit{Note:} This figure shows event study estimates
  comparing treated neighborhoods to matched controls. Each point represents
  the difference-in-differences estimate for the given year relative to
  public housing construction, with 95\% confidence intervals shown as shaded ribbons. The reference
  period is 10 years before the construction decade (event time = -10). The vertical dotted line indicates the timing of public housing construction.}
  \end{figure}



Figure \ref{fig:economic_housing} shows the effects of public housing on local neighborhood housing and labor market outcomes. 
Median rents fall by 0.03 log points immediately and reach -0.14 by 20 years and -0.36 by 40 years post-construction. These median rents include public housing residents, and likely reflect the fact that public housing rents were capped. 
Median income declines sharply, with log median income falling by 0.10 immediately upon construction and reaching -0.18 to -0.20 by 10-20 years post-construction, representing roughly an 18-20\%  decline in neighborhood median income.
Public housing neighborhoods also saw sizable increases in unemployment rates and falls in labor force participation rates, suggesting that public housing construction was associated with a decline in local economic activity.

  \begin{figure}[htbp]
      \centering
      \includegraphics[width=0.9\textwidth]{../output/regression_results/matched_did/combined/event_study_economic_housing_treated.pdf}
      \caption{Economic and Housing Effects in Treated Neighborhoods}
      \label{fig:economic_housing}
      \footnotesize{\textit{Note:} This figure shows event study estimates
  comparing treated neighborhoods to matched controls. Each point represents
  the difference-in-differences estimate for the given year relative to
  public housing construction, with 95\% confidence intervals shown as shaded ribbons. The reference period is 10 years before the construction decade (event time = -10).}
  \end{figure}


These results ultimately demonstate that the mid-century public housing program did not achieve its stated goals of neighborhood improvement and slum clearance, and instead concentrated poverty and segregation in the neighborhoods where it was built.
Even conditional on the fact that public housing was built in neighborhoods with higher Black population shares, lower incomes, and higher unemployment rates, the construction of these projects exacerbated these pre-existing disparities in the long-run. 
The results provide quantitative evidence supporting critics' claims that public housing contributed to neighborhood segregation and concentrated disadvantage. Rather than achieving the "slum clearance" and neighborhood improvement goals articulated in the 1949 Housing Act, public housing construction appears to have reinforced and extended patterns of racial and economic segregation in the neighborhoods where it was built.
Of course, many of these effects shown thus far are likely mechanical in nature, as the program grew to house the most disadvantaged populations. A more interesting question is to what degree these effects persisted in neighborhoods beyond the immediate vicinity of the public housing projects. In the following section, I attempt to answer that question.

\section{Spillover Effects}\label{sec:spillover_effects}
A primary source of backlash against the mid-century public housing program has been the concern it precipitated a broader urban decline (\cite{jacksonCrabgrassFrontierSuburbanization2006}), with negative spillovers on surrounding communities. 
To test this, I estimate the spillover effects of public housing on nearby neighborhoods.
First, I define the "spillover" neighborhoods as the set of census tracts that are contiguous to the treated neighborhoods.
Then, I apply the same k-nearest neighbors matched difference-in-differences approach as I do with the treated neighborhoods: For each spillover neighborhood, I identify a control neighborhood that was on a similar trajectory in the previous two decades. Balance statistics for the spillover neighborhoods are shown in Table \ref{tab:balance_spillover}.
\input{../output/balance_tables/matched_did/combined/balance_table_spillover_neighborhoods.tex}
\label{tab:balance_spillover}


I then estimate the treatment effects for these spillover neighborhoods using the same stacked difference-in-differences design as in Equation \ref{eq:matched_did}. 
Figures \ref{fig:spillover_population_demographics} and \ref{fig:spillover_economic_housing} present the spillover effects on
adjacent neighborhoods. Relative to the matched controls, I find that the nearby neighborhoods saw substantial long-run neighborhood changes as well,
though to a lesser extent than the treated neighborhoods. These neighborhoods saw an increase in black population share, driven by both increases in black population and long-run declines in white population, relative to their matched controls.
Spillover neighborhoods also saw substantial declines in median incomes, small increases in unemployment rates, and long-term decreases in median rents. 
These changes suggest that the effects of public housing extended beyond the immediate neighborhoods where they were built, contributing to broader patterns of urban decline and disinvestment.



  \begin{figure}[htbp]
      \centering
      \includegraphics[width=0.9\textwidth]{../output/regression_results/matched_did/combined/event_study_spillover_population_demographics.pdf}
      \caption{Spillover Effects: Population and Racial Composition in
  Adjacent Neighborhoods}
      \label{fig:spillover_population_demographics}
      \footnotesize{\textit{Note:} This figure shows event study estimates
  comparing adjacent neighborhoods (those contiguous to treated
  neighborhoods) to matched controls. Each point represents the
  difference-in-differences estimate for the given year relative to public
  housing construction, with 95\% confidence intervals shown as shaded
  ribbons. The reference period is 10 years before the construction decade (event time =-10). The vertical dotted line indicates the timing of public housing
  construction.}
  \end{figure}

  \begin{figure}[htbp]
    \centering
    \includegraphics[width=0.9\textwidth]{../output/regression_results/matched_did/combined/event_study_population_private_treated.pdf}
    \caption{Effects on Estimated Private Population, Treated Neighborhoods}
    \label{fig:private_population_treated}
    \footnotesize{\textit{Note:} This figure shows event study estimates for private population (excluding public
housing residents) comparing treated neighborhoods to matched controls. Private population is estimated by subtracting
public housing residents from total tract population. Each line represents the difference-in-differences estimate for
the given year relative to public housing construction, with 95\% confidence intervals shown as shaded ribbons. The
reference period is 10 years before the construction decade (event time = -10).}
\end{figure}


  \begin{figure}[htbp]
      \centering
      \includegraphics[width=0.9\textwidth]{../output/regression_results/matched_did/combined/event_study_spillover_economic_housing.pdf}
      \caption{Spillover Effects: Economic and Housing Outcomes in Adjacent
  Neighborhoods}
      \label{fig:spillover_economic_housing}
      \footnotesize{\textit{Note:} This figure shows event study estimates
  comparing adjacent neighborhoods (those contiguous to treated
  neighborhoods) to matched controls. Each point represents the
  difference-in-differences estimate for the given year relative to public
  housing construction, with 95\% confidence intervals shown as shaded
  ribbons. The reference period is 10 years before the construction decade (event time =
   -10). The vertical dotted line indicates the timing of public housing
  construction.}
  \end{figure}


%In the "inherited spillover approach", I find less evidence of these strong spillover effects. I do find similar declines in median incomes, and small declines in white population shares in the spillover neighborhoods. None of the other estimates, however, reach statistical signficance. The pre-treatment characteristics of these inherited spillover tracts, however, are less similar to the matched control group.

\section{Heterogeneity Analysis}\label{sec:heterogeneity}

In this section, I explore heterogeneity in the neighborhod effects of the public housing program. 






