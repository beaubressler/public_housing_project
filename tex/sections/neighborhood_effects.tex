\section{The Effect of Public Housing Construction on Neighborhoods}\label{sec:neighborhood_effects}
Having established in the previous section that public housing projects were systematically targeted towards poorer, minority neighborhoods, I now turn to estimating the effects of the projects on neighborhood change in subsequent decades.

\subsection{Research Design}\label{sec:research_design}
The central empirical challenge is to select appropriate counterfactual neighborhoods for those that received public housing.
To do so, I employ a \textbf{stacked matched difference-in-differences} approach informed by the site selection results in Section \ref{sec:site_selection_where}. 
I proceed as follows.

First, I identify a donor pool of potential control tracts for matching. 
I define these as the set of census tracts that never received public housing in my period and were never classified as nearby tracts, as defined in Section \ref{sec:data}.
Excluding these nearby neighborhoods from the donor pool helps avoid concerns about spillover effects.
In Section \ref{sec:spillover_effects}, I will directly test the effects of the construction of the projects on the nearby neighborhoods themselves.

Second, having identified a set of potential controls, I use propensity-score-based nearest neighbor matching with replacement to select comparison neighborhoods that are similar to the treated neighborhoods before public housing is constructed, specifically in terms of characteristics that predicted public housing placement.
In particular, I match on those pre-treatment characteristics which I found in Section \ref{sec:site_selection} were strongly statistically significant predictors of the placement of public housing projects: Total population, Black population share, median income, unemployment rate, labor force participation rate, median rent, and redlined status.
I require an exact match on redlined status, and whether a tract was an urban renewal tract, in order to avoid conflating the effects of these two policies.\footnote{While the site selection analysis in Section \ref{sec:site_selection_where} presented results using log transformations for population, incomes and rents, I match on levels to ensure treated and control neighborhoods are similar in absolute terms.}
My baseline specification also requires that the matched control be in the same county as the treated neighborhood.
%I also show in that results are similar when I require the matched control be in a different CBSA.
For each treated neighborhood, I identify the control with the closest propensity score.\footnote{Under these somewhat restrictive exact matching criteria, I do not find matches for 30 of the treated tracts.
I drop these neighborhoods from the analysis, but show that the results do not change if I loosen the criteria to allow matches with different redlined status.} 
Table \ref{tab:balance_treated} shows balance statistics for the treated neighborhoods and their matched controls, which shows fairly good balance on all pre-treatment characteristics. 
%The matching procedure achieves good balance on all pre-treatment characteristics, with standardized differences below 0.1 for all variables.

\input{../output/balance_tables/matched_did/combined/balance_table_treated_neighborhoods.tex}

Third, I create a stacked analytic dataset in which each treated neighborhood and its matched control appear for the full panel of years from 1930 to 2000. 
Each treated neighborhood and its matched pair is assigned the same treatment year and is identified with a matched pair identifiers.
Note that control neighborhoods may appear multiple times in the stacked dataset if they are matched to multiple treated neighborhoods.

Finally, I implement a stacked difference-in-differences design that compares changes in outcomes over time between each treated neighborhood and its matched control. 
In particular, for tract $i$ in matched pair $m$ at time $t$, I estimate the following event-study specification:

\begin{equation}\label{eq:matched_did}
y_{imt} = \alpha_i + \sum_{\tau\neq-10} \beta_\tau (D_{imt}^\tau \times \text{Treated}_i) + \sum_{\tau\neq-10} (\delta_m \times D_{imt}^\tau) + \varepsilon_{imt}
\end{equation}

where $y_{imt}$ is the outcome of interest for tract $i$ in matched pair $m$ at time $t$, $\alpha_i$ are tract fixed effects, $D_{imt}^\tau = \mathbb{1}[(t - T_{im}) = \tau]$ are indicators for event time $\tau$ relative to treatment year, $\text{Treated}_i$ indicates whether tract $i$ is a treated tract, and $\delta_m$ are matched pair fixed effects. 
By interacting each matched pair fixed effect is interacted with event time dummies, I ensure that each treated tract is explicitly compared to its matched control at each event time.
The $\beta_\tau$ terms measure the difference in outcomes between treated and control neighborhoods at each time period relative to the construction of public housing.

This event-study specification allows me to examine both pre-trends and dynamic treatment effects over time.\footnote{My balancing procedure ensures that I have at least one pre-trend estimate for each matched pair. Event times t=-30 and t=30 may not be observed for all matched pairs.}
Operationally, this specification is equivalent to a stacked difference-in-differences design in which each matched pair can be considered its own "sub-experiment" (\cite{wingStackedDifferenceDifferences2024}).
I weight each observation equally: since each treated unit is matched to a single control unit, one does not need to adjust for imbalances in treatment/control composition through weighting as outlined in \textcite{wingStackedDifferenceDifferences2024}.
By explicitly comparing matched pairs at each event time, this setup avoids econometric issues that have plagued the staggered adoption difference-in-differences setting (\cite{callawayDifferenceinDifferencesMultipleTime2021}).
Standard errors are adjusted for spatial correlation following \cite{conleyGMMEstimationCross1999} within a radius of 2 kilometers, allowing for dependence between nearby tracts.
Results are robust to using standard errors clustered at the tract level.

\subsection{Validity of the Research Design}
The key identification assumption is that, in the absence of public housing, the treated neighborhoods would have followed similar trends as their matched controls.
I cannot directly test this assumption, but I find no evidence of systematic pre-treatment trends in the two decades prior to construction.
Furthermore, funding for public housing was limited, and public housing authorities could not and did not build projects in every neighborhood that potentially could have received it.
As a result, many neighborhoods similar to those that received projects ultimately did not receive them, which creates plausible counterfactuals.
The benefit of matching to a neighborhood within the same county is that these control neighborhoods were subject to the same local political conditions and economic shocks.

On the other hand, one limitation with this restriction is that the quality of the matches may be limited by the availability of suitable controls within the county.
As a robustness check, I will  also show results where I require the matched control to be in a different CBSA, which increases the pool of potential controls and avoids concerns about within-county spillovers.
These results are shown in Appendix X.

%However, there remain several threats to identification, which I address below. 
%First, one potential concern would be if there were unobserved shocks that differentially affected the treated tracts.
%For example, if public housing was built in neighborhoods that saw an influx of Black residents in the years between t=-10 and t=0, the year that the project was built, the effects of public housing would be conflated with the effects of that shock.
%Furthermore, we might be concerned that neighborhoods that were chosen for public housing different on other unobservable characteristics from their matched controls.%\footnote{Other recent work on similar targeted policies has used proximity to the treatment to define the control group. For example, \textcite{blancoKnockingItMixing2025} estimate the effects of public housing regenerations in the UK on nearby neighborhoods by using slightly further neighborhoods/housing prices as a control group. In my setting, it is challenging to find a suitable control group purely using a distance-based approach that exhibits parallel pre-trend estimates. This may be due to the geographic size of Census tracts in some cities, especially in 1950. In Appendix \ref{sec:spatial_did_appendix}, I show estimates of the effects of public housing using a stacked spatial difference-in-differences approach, where the control tracts are defined purely by their proximity to the treated tracts.}

% TODO: Can fill this out

I will note that one could consider several other designs to estimate the effects of public housing construction.
One design would be to identify locations that were proposed for public housing but not ultimately selected.


For example, one could consider a distance-based stacked difference-in-differences design, where the control group is defined as tracts that are further away from the treated tracts, as in \textcite{blancoKnockingItMixing2025,guennewig-moenertPublicHousingDesign2024}.
I show in Appendix X that this design does not yield parallel pre-trends in my setting.
Another counterfactual could be to use the later-treated tracts as a control for the earlier-treated tracts, similar to the design in \textcite{deshpandeWhoScreenedOut2019}.
In Appendix X, I show that the later-treated tracts are systematically different from the earlier-treated tracts in my setting, 



\subsection{Effects on Treated Neighborhoods}\label{sec:effects_treated}
Figure \ref{fig:population_demographics_treated} presents the effects of public housing construction on population and racial composition in the treated neighborhoods.
The results show large and persistent demographic changes relative to the matched control neighborhoods.
Total population in the public housing neighborhoods increased substantially, with log population rising by $\approx$ 14\% in the immediate decade following construction.
This increase was driven by very large increases in the total Black population (56\%), with little effect on total white population on average. 
Relative to the control neighborhoods, Black population shares increased by 2.4 percentage points in the first decade following construction, and continued to increase up to approximately 4.2 percentage points in the third decade after construction (t=20). This represents a 15.5\% increase in the Black population share relative to the baseline share of 27.8\%.

Figure \ref{fig:private_population_treated} decomposes the population effects by examining changes in the estimated private population separately from the total population changes shown above.
This shows that the sizable increases in population were driven by increases in public housing population, which displaced residents in private housing.\footnote{As discussed in Section \ref{sec:data}, I do not have data on the population and racial composition for all projects, so these estimates are on a smaller sample of projects.}


  \begin{figure}[htbp]
      \centering
      \includegraphics[width=0.9\textwidth]{../output/regression_results/matched_did/combined/event_study_population_demographics_treated.pdf}
      \caption{Population and Racial Composition Effects in Treated
  Neighborhoods}
      \label{fig:population_demographics_treated}
      \footnotesize{\textit{Note:} This figure shows event study estimates
  comparing treated neighborhoods to matched controls. Each point represents
  the difference-in-differences estimate for the given year relative to
  public housing construction, with 95\% confidence intervals shown as shaded ribbons. The reference
  period is 10 years before the construction decade (event time = -10). The vertical dotted line indicates the timing of public housing construction.}
  \end{figure}

   \begin{figure}[htbp]
    \includegraphics[width=0.9\textwidth]{../output/regression_results/matched_did/combined/event_study_population_private_treated.pdf}
    \caption{Effects on Estimated Private Population, Treated Neighborhoods}
    \label{fig:private_population_treated}
    \footnotesize{\textit{Note:} This figure shows event study estimates for private population (excluding public
housing residents) comparing treated neighborhoods to matched controls. Private population is estimated by subtracting
public housing residents from total tract population. Each line represents the difference-in-differences estimate for
the given year relative to public housing construction, with 95\% confidence intervals shown as shaded ribbons. The
reference period is 10 years before the construction decade (event time = -10).}
\end{figure}


Figure \ref{fig:economic_housing} shows the effects of public housing on local neighborhood housing and labor market outcomes. 
Median rents fall by 0.03 log points immediately and reach -0.14 by 20 years and -0.36 by 40 years post-construction.
These median rents include public housing residents, and likely reflect the fact that public housing rents were capped. 
Median income declines sharply, with log median income falling by 0.10 immediately upon construction and reaching -0.18 to -0.20 by 10-20 years post-construction, representing roughly an 18-20\%  decline in neighborhood median income.
Public housing neighborhoods also saw sizable increases in unemployment rates and falls in labor force participation rates, suggesting that public housing construction was associated with a decline in local economic activity.

  \begin{figure}[htbp]
      \includegraphics[width=0.9\textwidth]{../output/regression_results/matched_did/combined/event_study_economic_housing_treated.pdf}
      \caption{Economic and Housing Effects in Treated Neighborhoods}
      \label{fig:economic_housing}
      \footnotesize{\textit{Note:} This figure shows event study estimates
  comparing treated neighborhoods to matched controls. Each point represents
  the difference-in-differences estimate for the given year relative to
  public housing construction, with 95\% confidence intervals shown as shaded ribbons. The reference period is 10 years before the construction decade (event time = -10).}
  \end{figure}


% TODO
%While interpretable, one possible concern is that within-county donor pool restriction may not provide an appropriate counterfactual if the remaining never-treated tracts in the same city differ systematically from the treated neighborhoods on unobserved dimensions.
%To address this, I create a new stacked data dataset by restricting the donor pool for each treated neighborhood to donor pool tracts only in \textit{other} CBSAs.
%I then estimate the same event-study specification as in Equation \ref{eq:matched_did} on this new dataset, but now also controlling for CBSA-by-year fixed effects.
%This specification can be thought of as comparing the deviation in outcomes of the treated neighborhoods from the CBSA in that year to the that of their matched control.

%Appendix Figure \ref{fig:outofcity_robustness} shows that the estimated dynamic effects are very similar to the main results shown in Figures \ref{fig:population_demographics_treated} and \ref{fig:economic_housing}.
%This similarity rules out the concern that the main results are driven by peculiarities of within-city donor scarcity or city-specific siting decisions by local housing authorities.

These results ultimately demonstrate that the mid-century public housing program did not achieve its stated goals of neighborhood improvement and slum clearance, and instead concentrated poverty and segregation in the neighborhoods where it was built.
Even conditional on the fact that public housing was built in neighborhoods with higher Black population shares, lower incomes, and higher unemployment rates, the construction of these projects exacerbated these pre-existing disparities in the long-run. 
The results provide quantitative evidence supporting critics' claims that public housing contributed to neighborhood segregation and concentrated disadvantage.
Rather than achieving the "slum clearance" and neighborhood improvement goals articulated in the 1949 Housing Act, public housing construction reinforced and extended patterns of racial and economic segregation in the neighborhoods where it was built.
Of course, many of these effects shown thus far are likely mechanical in nature, as the program grew to house the most disadvantaged populations.
A perhaps more interesting question is to what degree these effects persisted in neighborhoods beyond the immediate vicinity of the public housing projects.


\subsection{Effects on nearby neighborhoods}\label{sec:spillover_effects}
A primary source of backlash against the mid-century public housing program has been the concern it precipitated a broader urban decline and white flight, with negative spillovers on surrounding communities (\cite{jacksonCrabgrassFrontierSuburbanization1985}).
Understanding these potential spillovers is crucial for understanding the overall impact of the program. 
To test this, I estimate the geographic spillover effects of public housing on nearby neighborhoods.

As described in Section \ref{sec:data}, I define the nearby neighborhoods as the set of census tracts that are contiguous to the treated neighborhoods and are within 1 km of a public housing project.
Then, I apply the same nearest neighbor propensity score matching with replacement discussed in Section \ref{sec:research_design} to identify matched controls for these nearby neighborhoods.
Balance statistics for the nearby neighborhoods are shown in Table \ref{tab:balance_spillover}.

\input{../output/balance_tables/matched_did/combined/balance_table_spillover_neighborhoods.tex}
\label{tab:balance_spillover}

% TODO: Explain the results correctly 
I then estimate the treatment effects for these nearby neighborhoods using the same stacked difference-in-differences design as in Equation \ref{eq:matched_did}. 
Figures \ref{fig:spillover_population_demographics} and \ref{fig:spillover_economic_housing} present the effects on nearby neighborhoods.
I find relatively limited evidence of spillover effects on population and racial composition in the nearby neighborhoods.

Relative to the matched controls, I find that the nearby neighborhoods to public housing saw some, but relatively limited neighborhood and housing market changes on average.
In particular, I estimate statistically significant declines in median income in the short- and long-run.
I find potential evidence of small long-run declines total population and white population, and small increases in the Black population share in these nearby neighborhoods as well, though these results do not reach statistical significance.
Ultimately, this evidence suggests that the geographic spillovers of public housing construction on nearby neighborhoods were relatively limited, and that the most pronounced effects of the program were largely concentrated in the neighborhoods where the projects were built.





  \begin{figure}[htbp]
      \includegraphics[width=0.9\textwidth]{../output/regression_results/matched_did/combined/event_study_spillover_population_demographics.pdf}
      \caption{Spillover Effects: Population and Racial Composition in Adjacent Neighborhoods}
      \label{fig:spillover_population_demographics}
      \footnotesize{\textit{Note:} This figure shows event study estimates
  comparing adjacent neighborhoods (those contiguous to treated
  neighborhoods) to matched controls. Each point represents the
  difference-in-differences estimate for the given year relative to public
  housing construction, with 95\% confidence intervals shown as shaded
  ribbons. The reference period is 10 years before the construction decade (event time =-10). The vertical dotted line indicates the timing of public housing
  construction.}
  \end{figure}

  \begin{figure}[htbp]
      \includegraphics[width=0.9\textwidth]{../output/regression_results/matched_did/combined/event_study_spillover_economic_housing.pdf}
      \caption{Spillover Effects: Economic and Housing Outcomes in Adjacent Neighborhoods}
      \label{fig:spillover_economic_housing}
      \footnotesize{\textit{Note:} This figure shows event study estimates comparing adjacent neighborhoods (those contiguous to treated
  neighborhoods) to matched controls. Each point represents the difference-in-differences estimate for the given year relative to public
  housing construction, with 95\% confidence intervals shown as shaded ribbons. The reference period is 10 years before the construction decade (event time =
   -10). The vertical dotted line indicates the timing of public housing construction.}
  \end{figure}


%In the "inherited spillover approach", I find less evidence of these strong spillover effects. I do find similar declines in median incomes, and small declines in white population shares in the spillover neighborhoods. None of the other estimates, however, reach statistical signficance. The pre-treatment characteristics of these inherited spillover tracts, however, are less similar to the matched control group.

\section{Heterogeneity Analysis}\label{sec:heterogeneity}

In this section, I explore heterogeneity in the neighborhood effects of the public housing program.
For simplicity, I show estimates at t=20, or 20 years after the first decade of public housing construction, which capture the long-run effects of the program.
There are several dimensions along which the effects of public housing may have varied.

First, the effects may have varied based on the initial neighborhood characteristics.
Modern evidence on the construction of affordable housing projects through the Low Income Housing Tax Credit (LIHTC) program suggests that whether the new projects are a local amenity or disamenity depends on the characteristics of the neighborhood (\cite{diamondWhoWantsAffordable2019}).
Furthermore, models of neighborhood tipping suggest that the initial racial composition of the neighborhood may influence the extent to which public housing construction precipitates white flight or overall racial transition (\cite{schellingDynamicModelsSegregation1971,cardTippingDynamicsSegregation2008}).
Some historical accounts of public housing argue that public housing construction may have led to neighborhood racial transition through tipping dynamics (\cite{rothsteinColorLawForgotten2017,jacksonCrabgrassFrontierSuburbanization1985}).

To test this tipping hypothesis, I examine how the effects of public housing varied based on the initial Black population share of the neighborhood.
I divide treated and nearby neighborhoods into three groups based on their baseline Black population share measured 10 years before public housing construction:
Neighborhoods that are almost entirely non-Black (less than 1\% Black share), those with "medium" initial Black shares that I will consider as those in the tipping range (between 1\% and 12\%) and those with high initial Black shares (12\% or higher).
I choose 12\% for the top of the "medium" range, as it is the top of the tipping threshold estimated by \textcite{cardTippingDynamicsSegregation2008} in 1970. 

Figure \ref{fig:baseline_black_share_heterogeneity} presents the results of this analysis at time t=20.
I show estimates using raw population counts rather than log population, since percentage changes in Black population will mechanically be larger in neighborhoods with small initial Black populations.
I find large differences in the effects of public housing on the racial composition of neighborhoods based on their initial Black population share.
Treated neighborhoods that were almost entirely non-Black and those in the "tipping range" saw sizable increases in Black population in responses to public housing construction, while those with high initial Black shares saw no change in Black population, and if anything, some increase in white population.
Similarly, nearby neighborhoods in the "tipping range" also saw sizable declines in white population, potentially suggesting some degree of white flight in response to public housing construction for neighborhoods around this range. 

\begin{figure}[htbp]
\includegraphics[width=0.9\textwidth]{../output/regression_results/baseline_black_share_heterogeneity_t20.pdf}
\caption{Heterogeneity by Initial Neighborhood Racial Composition}
\label{fig:baseline_black_share_heterogeneity}
\footnotesize{\textit{Note:} This figure compares the effects of public housing construction in neighborhoods with
  very low (less than 1\%), medium (1-12\%) and high (greater than 12\%) baseline Black shares. Baseline is
measured at t=-10. The top panels show effects on treated neighborhoods, and bottom
panels show spillover effects on nearby neighborhoods.}
\end{figure}


I also test whether the effects of the projects varied based on when the projects were built.
Comparing the effects of the projects built in the 1940s and 1950s to those built after 1960 may highlight how the changing political and social context surrounding public housing influenced its neighborhood effects.
Figure \ref{fig:construction_decade_heterogeneity} shows the results of this analysis at t=20.
I find that, particularly for population and racial composition, the effects differed substantially based on when the projects were built.
In particular, I see evidence of substantial long-run Black population increase in public housing neighborhoods, and long-run white population exit in nearby neighborhoods for projects built in the early period, but not for those built later.
I am still exploring the mechanisms behind this result.
One possibility is that this reflects the changing racial composition of American cities over this period and reflects the same racial tipping mechanisms as in Figure \ref{fig:baseline_black_share_heterogeneity}:
For projects built earlier on, the Black population in the average neighborhood was much lower.  

  \begin{figure}[htbp]
  \includegraphics[width=0.9\textwidth]{../output/regression_results/early_vs_late_programs_race_income_t20.pdf}
  \caption{Long-Run Effects of Early vs Later Public Housing Projects}
  \label{fig:early_vs_late_programs_race_income_t20}
  \footnotesize{\textit{Note:} This figure compares the effects of public housing construction in neighborhoods with
  very low (less than 1\%), medium (1-12\%) and high (greater than 12\%) baseline Black shares. Baseline is
measured at t=-10. The top panels show effects on treated neighborhoods, and bottom
panels show spillover effects on nearby neighborhoods.}
  \end{figure}


Finally, I test whether these effects differed in neighborhoods that were also affected by the urban renewal program.
This analysis both highlights potential interactions between the two programs and addresses concerns that my baseline estimates may be conflated with the effects of urban renewal.
Figure \ref{fig:urban_renewal_heterogeneity} presents the results in t=20, separately estimating Equation \ref{eq:matched_did} for neighborhoods that were urban renewal tracts and those that were not.
I find null effects on median income, and population by race in neighborhoods that were also urban renewal tracts, suggesting that the impact of urban renewal may have outweighed any effects of public housing.
In contrast, in neighborhoods not targeted by urban renewal, I find similar effects as in the main analysis.
%One interesting finding is that in urban renewal tracts that were nearby public housing, I find sharp declines in long-run Black population.
%This is consistent with the historical narrative that public housing often served to house individuals displaced by urban renewal (\cite{baumanPublicHousingDreadful1994, hirschMakingSecondGhetto1998}), and may reflect the fact that urban renewal displaced many Black residents from these neighborhoods.

  \begin{figure}[htbp]
  \includegraphics[width=0.9\textwidth]{../output/regression_results/urban_renewal_heterogeneity_race_income_t20.pdf}
  \caption{Long-Run Effects of Early vs Later Public Housing Projects}
  \label{fig:urban_renewal_heterogeneity}
  \end{figure}


