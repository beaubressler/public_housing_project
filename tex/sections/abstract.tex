This paper examines the long-term neighborhood effects of the American public housing program, one of the largest and most controversial American urban policies of the 20th century.
I construct a new national dataset tracking the locations, completion dates, and characteristics of over 7,000 public housing projects built between 1935 and 1973, and combine it with eight decades of census tract-level data to examine how public housing construction affected neighborhood trajectories.
I first show that public housing projects were systematically targeted toward neighborhoods that were initially poorer, more populated, and had higher shares of Black residents, reflecting the program's slum clearance goals and the racialized politics of 20th-century housing policy.
Using a stacked matched difference-in-differences approach, I then estimate the causal effects of public housing construction on neighborhood change, comparing treated neighborhoods to matched controls within the same county, based on the characteristics that predict public housing placement.
Neighborhoods receiving public housing experienced large and persistent increases in Black population and population shares, along with substantial declines in median incomes and other socioeconomic indicators.
Geographic spillovers to nearby neighborhoods were more limited: incomes declined but demographic composition showed little change on average.
The findings demonstrate that, despite intentions of slum clearance and neighborhood revitalization, public housing reinforced existing patterns of economic and racial segregation. However, these effects were largely confined to the project neighborhoods, suggesting that public housing did not precipitate the broad urban decline that some narratives suggest. 
%The findings suggest that despite intentions of slum clearance and neighborhood revitalization, public housing construction contributed to long-term neighborhood demographic changes that reinforced and potentially amplified patterns of economic and racial segregation in American cities.
