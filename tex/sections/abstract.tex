This paper studies the long-run neighborhood effects of one of the most ambitious U.S. urban policies in 20th century: the construction of public housing projects. I build a new national dataset tracking the locations, completion dates, and characteristics of over 7,000 public housing projects built from 1935 to 1973, and link it to 70 years of neighborhood-level census data. I first document that public housing was disproportionately built in redlined neighborhoods with higher Black population shares, higher unemployment rates, and lower median incomes. Then, using a matched difference-in-differences approach, I show that neighborhoods receiving public housing experienced increases in total population and Black population shares, coupled with declining socioeconomic status. Nearby neighborhoods saw decreases in property values, rents, and socioeconomic status of residents. These spillover effects were stronger for neighborhoods near more public housing development. The findings suggest that despite intentions of slum clearance and neighborhood revitalization, public housing construction contributed to long-term neighborhood transformation that reinforced patterns of economic and racial segregation in American cities.
