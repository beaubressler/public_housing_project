This paper studies the long-run neighborhood effects of one of the largest and most controvertial American urban policies of 20th century: the public housing program.
I build a new national dataset tracking the locations, completion dates, and characteristics of over 7,000 public housing projects built from 1935 to 1973 combined with eight decades of neighborhood characteristics. 
I first document the systemic targeting of public housing projects towards poorer, minority neighborhoods, particularly those that had been redlined in the 1930s.
Then, using a matched difference-in-differences approach, I estimate the long-run effects of public housing construction on neighborhood change, both in the neighborhoods that received public housing and in nearby neighborhoods.
I find that neighborhoods receiving public housing experienced large long-run increases in Black population and population shares, coupled with declines in median incomes and other socioeconomic indicators.
These projects also led to long-term declines in median incomes and labor force participation in surrounding neighborhoods.
The findings suggest that despite intentions of slum clearance and neighborhood revitalization, public housing construction contributed to long-term neighborhood transformation that reinforced patterns of economic and racial segregation in American cities.
