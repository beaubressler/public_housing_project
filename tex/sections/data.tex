\section{Data}\label{sec:data}

\subsection{Public Housing Data}

A significant challenge in studying the history of public housing is the absence of a comprehensive dataset containing the list of projects, construction dates, and precise locations of the projects. Consequently, previous research on the neighborhood effects of public housing has been limited in scope or restricted to a small number of cities where researchers could obtain this information directly from housing authorities. My paper addresses this gap by constructing a more complete dataset, combining information from public, recently digitized, and newly digitized sources.

The first source I use is the publicly available Picture of Subsidized Households (PSH) datasets from the US Department of Housing and Urban Development (HUD).
These datasets contain a list of federally-funded housing projects, various characteristics of the project (e.g., number of units, demographics) and, from 1997 onward, their locations.
I use the PSH datasets from 1997, 2000, and 2023, due to gaps in each of them.
Crucially, however, these data do not contain the construction date of the projects, and therefore are not sufficient by themselves to study the effects of construction of the projects.

To remedy this, I merge these data with the  \textit{Consolidated Development Directory} (CDD), published by HUD in 1973 and digitized by \textcite{shesterLocalEconomicEffects2013}.
This data contains the universe federally-funded public housing projects which existed in 1973, along with various project characteristics (e.g. number of units) and, crucially, the year each project was completed.
Both the PSH datasets and the CDD contain HUD-assigned project codes, enabling linkage between the two datasets together.
This linkage allows me to assign locations to over 90\% of the projects in the CDD.
%I explain the linkage procedure in Appendix \ref{sec:appendix_A}.

I supplemented these data with hand-collected information from historical annual reports of local public housing agencies, obtained from various libraries (or directly from the housing authority, in the case of San Francisco), and data obtained from FOIA requests of public housing agencies.
From this effort, I was able to obtain supplementary data from seven major cities:
New York, Chicago, Los Angeles, Washington DC, San Francisco, Atlanta, and Baltimore.
For these cities, I collected the full set of projects built up to 1973, including their construction dates and locations.
I was able to geolocate these projects using the Google Maps API.
For projects for that I was unable to successfully geolocate, I manually filled in the locations.
%More details on the geolocation procedure can be found in the Appendix. 

There were two motivations for collecting these additional data.
First, it allowed for a better understanding of the projects in the CDD that I was unable to match to the PSH.
Public housing demolitions due to HOPE VI, which began in 1993, and earlier demolitions meant that some projects in the CDD might be missing from the PSH datasets.
Indeed, the incomplete matching between CDD and PSH data suggests that some projects may have ceased to when PSH data collection began.
Moreover, these missing projects might not be randomly distributed, as demolitions targeted particularly blighted projects. 

Second, the CDD-HUD data only include federally funded public housing projects. However, some cities had significant city- and state-funded programs.
New York City, in particular, has a notable city- and state-funded public housing program, and using only the data on federal projects in New York City misses a substantial number of housing projects and units.\footnote{My digitized dataset of NYC housing projects contains 129,430 housing units, compared to 80,000 in the PSH-CDD data.}
The other cities for which I collected data had either none or few city or state-funded projects.
Ultimately, I use my hand-collected data for housing projects in New York City, Chicago, and San Francisco, supplement the PSH-CDD for Boston and Washington DC, and rely on the PSH-CDD data for all other cities.\footnote{For the other cities for which I hand-collected data, I find no additional projects in my hand-collected data versus the PSH-CDD data.}
I found little evidence in the historical record that by otherwise relying on data on federal projects, I am missing a notable stock of public housing in other cities. 

The last step in constructing the public housing dataset is to obtain information on the populations and racial composition of the projects. 
For most cities, I obtain this information from the 1977 Picture of Subsidized Households dataset, which was cleaned and provided to me by Yana Kucheva (\cite{kuchevaSubsidizedHousingConcentration2013}).
Here again, I was able to link these data via federal project codes.
For New York City, I used population-by-race data from the early 1970s provided to me by Max Guennewig-Moenert (\cite{guennewig-moenertPublicHousingDesign2024}).
And for Chicago, I obtained these data from the 1973 digitized Annual Report of the Chicago Housing Authority. 

The end result of this process is a dataset containing the construction dates, locations, and characteristics of over 1 million units of public housing built from 1935 until 1973.

\subsection{Neighborhood Data}
In order to study the neighborhood effects of public housing construction, I construct a panel dataset of neighborhood-level characteristics from 1930 to 2000.
Building such a dataset presents several challenges. First, the number of cities for which tract-level data exist is limited in 1940 and especially 1930.
Second, Census tract boundaries change over time, requiring the construction of a consistent panel over time.
And third, income data was not collected in the 1930 Census, and median income was not reported in the publicly available tract-level Census tables in 1940.

I proceed as follows. 
First, I collect outcomes at the Census tract-level from the 1930 to 2000 decennial censuses.
These data include tract-level measures of population by race (white and Black), several socioeconomic measures (median income, high school graduation rate, labor force participation and unemployment rates), and median rents and home values.
All census data and shapefiles were acquired from IPUMS NHGIS (\cite{mansonNationalHistoricalGeographic2022}). 
 
I construct a consistent panel of census tracts by concording all tracts to 1950 Census tract boundaries using an area-reweighting approach, following \textcite{eckertMethodConstructGeographical2020}.
I choose 1950 as the base year for several reasons. 
First, in order to limit concerns about public housing-driven changes in tract boundaries, I choose a year early in my analysis period, rather than at the end, as much of the literature does.
Second, publicly available 1940 New York City census tract shapefiles do not correspond to actual Census tracts, but much larger health districts, making 1940 a less suitable base year.
%In Appendix X, I show that my results are robust to using 2010 tract boundaries as the base year, using the tract crosswalk from \textcite{leeNaturalAmenitiesNeighbourhood2018}.

I supplement this tract-level data with data from the 1930 and 1940 full-count Census (\cite{rugglesIPUMSUSAVersion2022}).
To convert this to the Census tract level, I merge the full count data to the 1930 and 1940 enumeration district shapefiles provided by the Urban Transitions Project (\cite{loganSideStreetGhetto2024}) and collapse these data to the enumeration district level.
I then, again, use area-reweighting to aggregate these data to 1950 Census tracts.
This process both allows me to expand my tract-level dataset to cities and neighborhoods for which tract-level data was unavailable in 1930 and 1940, as well as include 1930 and 1940 income information, which is not available in the tract-level tables. 
I proxy for income in 1940 using total wage income in each household, and in 1930 using machine-learning adjusted occupation scores from \textcite{saavedraMachineLearningApproach2020}.
I convert all monetary values to 2000 dollar values using US CPI from \textcite{officerMeasuringWorth2025}.

%TODO:
I also incorporate the data and shapefiles of the Home Owners' Loan Corporation (HOLC) ``redlining'' maps drawn in the late 1930s, made available by \textit{Mapping Inequality} (\cite{nelsonMappingInequalityRedlining2023}).
These maps graded neighborhoods from ``A'' (best) to ``D'' (hazardous) to indicate perceived mortgage risk.
Areas with high minority populations were systematically marked as ``hazardous'' (red), which some have argued may have limited mortgage lending and investment and contributed to segregation (\cite{aaronsonEffects1930sHOLC2021}).
More recent research, however, suggests that these maps did not directly shape HOLC lending decisions and instead primarily reflected existing segregation patterns rather than creating them (\cite{fishbackHOLCMapsHow2020,fishbackNewEvidenceRedlining2024}).
For my purposes, these maps provide a measure of neighborhood racial composition and socioeconomic status in the late 1930s and serve as a proxy for existing discrimination.
I overlay these maps onto the 1950 Census tract boundaries. Following \textcite{weiwuUnequalAccessRacial2025}, I classify a neighborhood as ``redlined" if 80\% of the area is designated as ``hazardous'' (HOLC grade D).

Finally, I incorporate data on the locations of U.S. urban renewal projects (1955-1966) from \textit{Renewing Inequality} (\cite{nelsonRenewingInequality2025}). 
Urban renewal, established under Title I of the 1949 Housing Act, provided federal subsidies for cities to acquire and clear ``blighted'' neighborhoods. 
The program was closely connected to public housing: clearance projects often created sites for new developments or displaced families that public housing was intended to rehouse (\cite{vonhoffmanStudyContradictionsOrigins2000}).
Recent research finds that Black neighborhoods were two to three times more likely to be cleared than white neighborhoods, and that treated areas experienced declines in housing and population density alongside rising rents and incomes (\cite{lavoiceLongrunImplicationsSlum2024}).
I include these locations to account for potentially confounding effects of urban renewal and to explore the relationship between the two programs.
I overlay these maps onto the 1950 Census tract boundaries and classify a tract as an ``urban renewal'' tract if more than 5\% of its area was designated for urban renewal.


\subsection{Treatment Definition}

I define a tract as treated if any portion of the tract lies within a 50-meter buffer of a public housing project.
This definition captures scenarios in which a project is built on the border of multiple tracts, and thus may span multiple tracts.\footnote{In cases in which a project spans multiple tracts, I divide the project's units and population evenly across the affected tracts.}
To ensure that the public housing project represents a potentially meaningful neighborhood intervention, I exclude projects that contain fewer than 50 housing units.

In terms of treatment timing, because my Census data are decennial, a tract is considered treated in year $t$ if it receives its first public housing project between years $t-9$ and $t$ (e.g., tracts receiving projects between 1951-1960 are treated in 1960).
This timing convention ensures that the treatment occurs after the pre-treatment observation but before the post-treatment observation in the panel.
\footnote{I do not have the exact construction dates for all projects, only the year of completion.
In the case in which a project was completed in the Census year but after the Census enumeration date, the effects of the project would meaningfully be captured in the second, rather than first, post-treatment decade, and the $t=0$ effects would be understated.}
For tracts receiving multiple public housing projects over time, treatment timing is determined by the first project meeting the size thresholds.
Subsequent projects in the same tract are not considered separate treatment events, as the tract has already been "treated" by public housing intervention.

\subsection{Spillover Definition}
To study geographic spillovers, I define a tract as being a "nearby" tract if it is contiguous to a treated tract, is not itself a treated tract, and is within 1 kilometer of the nearest public housing project.
This hybrid contiguity-and-distance definition captures neighborhoods that are close enough to potentially be affected by the public housing project, while excluding tracts that are technically adjacent but far away from the project due to, for example, irregular boundaries. 
I choose 1 kilometer as a distance threshold based on previous literature showing the geographic extent of spatial spillovers from public housing interventions (\cite{blancoKnockingItMixing2025}).
For timing, spillover tracts inherit the earliest treatment year of their adjacent treated tracts.
Figure \ref{fig:chicago_spillover} illustrates this spillover definition using Chicago as an example, showing public housing projects (treated tracts) and the adjacent spillover areas.



 \subsection{Sample Selection}
My empirical analysis focuses on projects built between 1941 and 1973, with neighborhood outcomes measured from 1930 to 2000.
I restrict the sample in several ways to ensure that I have a balanced panel of neighborhoods with sufficient pre-treatment data to estimate the effects of public housing construction.

Table \ref{tab:sample_attrition} shows the impact of each filtering step on the sample composition.
Starting from the original sample of 12,062 census tracts representing 38.1 million people in 1940 (approximately 29\% of the U.S. population), I apply the following restrictions.
First, I require that tracts exist in all years from 1930 to 2000, which drops 25\% of tracts, primarily due to new tracts created in later decades or areas that lacked tract coverage in early years.
Second, I drop tracts for which I cannot calculate population by race, median income, median rent, labor force participation rates, and unemployment rates for all years, which removes an additional 5\% of tracts.
Third, I exclude tracts with public housing built before 1941 to preserve 1930 and 1940 as clean pre-treatment periods, which removes 91 treated tracts containing 70 projects.
Fourth, I drop metropolitan areas with fewer than 30 total tracts to ensure sufficient within-city variation.
Finally, I exclude outlier tracts with populations below 500 or above the 98th percentile (14,651) in any year, and I drop CBSAs where more than 30\% of tracts are treated, to ensure credible control groups.

The resulting balanced sample includes 6,430 census tracts across 45 CBSAs, representing 27.2 million people in 1940 (approximately 21\% of the U.S. population).
The sample contains 640 public housing projects with 265,505 total units located in 719 treated tracts.
Overall, the filtering process retains 53\% of the original census tracts and 55\% of the original public housing projects, with the largest attrition occurring from the balanced panel requirements and population outlier exclusions.
Figure \ref{fig:cbsa_coverage} shows the geographic distribution of CBSAs included in the analysis.



\subsection{Summary Statistics}

Table \ref{tab:tract_summary} shows summary statistics for census tract characteristics in 1940 and 1990.

