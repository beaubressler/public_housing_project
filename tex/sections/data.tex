\section{Data}\label{sec:data}

\subsection{Public Housing Data}

A significant challenge in studying the history of public housing is the lack of a comprehensive dataset that includes project lists, construction dates, and precise project locations. Consequently, previous research on the neighborhood effects of public housing has been limited in scope or restricted to a small number of cities where researchers could obtain this information directly from housing authorities.
The data limitations have also hindered broader historical analyses of the public housing program, and have been acknowledged by previous work \parencite{ellenDoesFederallySubsidized2007,huntPublicHousingUrban2018}.
My paper addresses this gap by constructing what I believe to be the most complete dataset of mid-century public housing by combining information from a series of administrative datasets, along with previously digitized and newly digitized sources.

The first source I use is the \textit{Consolidated Development Directory} (CDD), published by the US Department of Housing and Urban Development (HUD) in 1973 and digitized by \textcite{shesterLocalEconomicEffects2013}.
This data contains the universe of federally funded public housing projects that existed in 1973, along with various project characteristics (e.g., number of units) and, crucially, the year each project was completed.
However, the CDD does not contain location information for the projects.
Previous work using these data \parencite{shesterLocalEconomicEffects2013,shesterConcreteMeasuresRise2019} has therefore been limited to studying aggregate city- and county-level outcomes.
These data also contain three project number fields, which I combine with the state code to form federal project codes in the standard HUD format (e.g., IL-2-38-1 for a Chicago project).
These project codes allow me to link the CDD data to other project-level datasets such as the Picture of Subsidized Households and HUD Form 951.
As far as I know, this element of these data has not been previously exploited.

To obtain location information for the projects, I turn to a series of publicly available HUD administrative datasets.
The primary of these is the Picture of Subsidized Households (PSH) datasets, which contain a list of federally funded housing projects, various project characteristics (e.g., number of units, demographics), and, from 1997 onward, their locations.
I use the PSH datasets from 2000 and 1997, due to limitations in each. 
I also link to more recent National Data Geospatial Asset (2023) data.
Finally, I link to the HUD File 951 dataset, which lists street addresses, latitudes, and longitudes for the stock of multifamily assisted housing projects that existed between 1986 and 1995 \parencite{kuchevaSubsidizedHousingConcentration2013}.
These datasets largely overlap in coverage, but each contains projects missing from the others.
This linkage allows me to assign locations to over 90\% of the projects in the CDD.
I will refer to this as the geocoded CDD dataset. 

%I explain the linkage procedure in Appendix \ref{sec:appendix_A}.

I supplemented these data with hand-collected information from historical annual reports of local public housing agencies, obtained from various libraries (or directly from the housing authority in the case of San Francisco), and from FOIA requests to public housing agencies.
From this effort, I was able to obtain supplementary data from eight major cities:
New York, Chicago, Boston, Los Angeles, Washington, DC, San Francisco, Atlanta, and Baltimore.\footnote{I collected digitized reports from other cities, such as Cleveland, Cincinnati, and St. Louis, but not all contained the precise project locations information.}
For these cities, I collected the complete set of projects built up to 1973, including their construction dates and locations.
I was able to geolocate these projects using the Google Maps API.
For projects that I was unable to geolocate successfully, I hand-identified locations based on street address and name.
%More details on the geolocation procedure can be found in the Appendix. 

There were two motivations for collecting these additional data.
First, it allowed for a better understanding of the projects in the CDD for which I was unable to assign locations.
Public housing demolitions, such as those from the HOPE VI program that began in 1993, may have caused some CDD projects to be absent from later administrative datasets used for geocoding, preventing me from assigning locations to those projects in the merged dataset.
Indeed, the incomplete matching between CDD and PSH data suggests that some projects may have ceased to exist when PSH data collection began.
Moreover, these missing projects might not be randomly distributed, as demolitions targeted particularly blighted projects.

Second, the CDD-HUD data include only federally funded public housing projects.
Little data exist on city- and state-funded public housing projects, but some cities may have funded some public housing construction outside of the federal program.
New York City, in particular, has a notable city- and state-funded public housing program, and using only the data on federal projects in New York City misses a substantial number of housing projects and units.\footnote{My digitized dataset of NYC housing projects contains 129,430 housing units, compared to 80,000 in the PSH-CDD data.}
The other cities for which I collected data had either none or very few city or state-funded projects.
Ultimately, I use my hand-collected data for housing projects in New York City, Chicago, and San Francisco, supplement the geocoded CDD for Boston and Washington, DC, and rely on the geocoded CDD data for all other cities (Los Angeles, Atlanta, and Baltimore).\footnote{For Los Angeles, Atlanta, and Baltimore, I find no additional projects in my digitized data versus the geocoded CDD data.}
I found little evidence in the historical record suggesting that, by otherwise relying on data on federal projects, I am missing a notable stock of public housing in other cities. 

The last step in constructing the public housing dataset is to obtain information on the populations and racial composition of the projects.
These data allow me to distinguish between public housing residents and the rest of the neighborhood's population.
For most cities, I obtain this information from the 1977 Picture of Subsidized Households dataset, which was cleaned and shared with me by Yana Kucheva \parencite{kuchevaSubsidizedHousingConcentration2013}.
% TODO: Explain how you get 
These data report the number of subsidized households by race in each project, but do not include direct population counts. 
In Appendix \ref{app:public_housing_data}, I describe how I convert these household counts to population counts.
I match these data to the geocoded CDD dataset using federal project codes.
For New York City, I used population-by-race data from the early 1970s digitized and shared with me by Max Guennewig-Moenert \parencite{guennewig-moenertPublicHousingPreferences2025}.
And for Chicago, I obtained population-by-race data from the 1973 digitized Annual Report of the Chicago Housing Authority.
While it would be ideal to measure population by race at the time of construction, these data are not available in most cities.
Thus, I assume the population and racial composition of each project in each decade after construction is equal to the 1970s values.
%For the panel analysis, I assign these 1970s population estimates to the treatment year and all subsequent decades, assuming they remain constant over time.
%For pre-treatment years, the public housing population is set to zero.
I use these population estimates to estimate the private (non-public-housing) population in each decade by subtracting the estimated public-housing population from the total Census-reported population.

The result of this process is a dataset containing the construction dates, coordinates, and characteristics of over 8,000 projects and 1 million units of public housing built from 1935 until 1973. 

\subsection{Neighborhood Data}\label{sec:data_neighborhoods}
To study the neighborhood effects of public housing construction, I construct a panel dataset of neighborhood-level characteristics spanning 1930 to 2010.
Building such a dataset presents several challenges.
First, the number of cities with tract-level data is limited in 1940 and especially in 1930.
Second, Census tract boundaries change over time, requiring the construction of a consistent panel.
And third, income data was not collected in the 1930 Census, and median income was not reported in the tract-level Census tables in 1940.

I proceed as follows. 
First, I collect a variety of outcomes at the Census tract level from the 1930 to 2010 decennial censuses.
These data include tract-level population counts by race (white, Black, and other), several socioeconomic measures (median income, high school graduation rates, labor force participation, and unemployment rates), and median rents and home values.
All census data and shapefiles were acquired from IPUMS NHGIS \parencite{mansonNationalHistoricalGeographic2022}. 
 
I construct a consistent panel of census tracts by concording all tracts to 1950 Census tract boundaries using an area-reweighting approach. 
I describe the tract harmonization in more detail in Appendix \ref{app:tract_harmonization}.
I chose 1950 as the base year for two reasons. 
First, to limit concerns about results being driven by public housing-driven changes in tract boundaries, I chose a year early in my analysis period, rather than at the end, as much of the literature does.
Second, publicly available 1940 New York City census tract shapefiles do not correspond to actual census tracts, but to much larger health districts, making 1940 a less suitable base year.
Given that New York City is a major city in my sample, I chose 1950 as the base year for the entire sample.
%In Appendix X, I show that my results are robust to using 2010 tract boundaries as the base year, using the tract crosswalk from \textcite{leeNaturalAmenitiesNeighbourhood2018}.

I supplement this tract-level data with data from the 1930 and 1940 full-count Census \parencite{rugglesIPUMSUSAVersion2022}.
To convert the full count data to the tract level, I first aggregate the individual-level data to the enumeration district (ED) level using the 1930 and 1940 enumeration district shapefiles from the Urban Transitions Project \parencite{loganSideStreetGhetto2024}.
I then use area reweighting to aggregate these data to 1950 Census tracts.
Incorporating these full-count data serves two purposes. 
First, they allow me to expand my set of cities and neighborhoods for which tract-level data was unavailable in 1930 and 1940.
Second, they allow me to include income information for 1930 and 1940, which is not available in the tract-level tables for those years. 
I proxy for income in 1940 and 1930 using full-count Census microdata.
For 1940, I sum individual wage income (INCWAGE) across all household members to calculate total household wage income, then classify households into income bins within each enumeration district.
For 1930, I use machine-learning-adjusted occupation scores from \textcite{saavedraMachineLearningApproach2020} aggregated to the household level and similarly binned.
In both years, I use income bins that match the 1950 census income categories.
I concord these enumeration district-level data to 1950 Census tracts using area-reweighting and calculate tract-level medians from the binned distributions.
The area-reweighting and median calculation procedures are described in Appendix \ref{app:tract_harmonization}.
I convert all monetary values to 2000-dollar values using the US CPI from \textcite{officerMeasuringWorth2025}.

In addition to Census data, I incorporate several historical datasets that capture other urban policies that be related to public housing placement and subsequent neighborhood change.
First, I incorporate digitized data and shapefiles of the Home Owners' Loan Corporation (HOLC) ``redlining'' maps drawn in the late 1930s, made available by \textit{Mapping Inequality} \parencite{nelsonMappingInequalityRedlining2023}.
These maps graded neighborhoods from ``A'' (best) to ``D'' (hazardous) to indicate perceived mortgage risk.
Although many scholars have argued that these maps institutionalized redlining and curtailed investment in minority neighborhoods \parencite{aaronsonEffects1930sHOLC2021}, recent evidence complicates this view.
\textcite{fishbackNewEvidenceRedlining2024} shows that the maps largely reflected existing patterns of racial and socioeconomic segregation rather than shaping new lending decisions.
For my purposes, these maps serve as a historical snapshot of neighborhood conditions and perceived credit risk in the late 1930s and thus serve as a proxy for pre-existing discrimination, rather than a direct driver of exclusion.
I overlay these maps onto the 1950 Census tract boundaries. Following \textcite{weiwuUnequalAccessRacial2025}, I classify a neighborhood as ``redlined" if at least 80\% of the area is designated as ``hazardous'' (HOLC grade D).

Second, I also incorporate data on the locations of U.S. urban renewal projects (1955-1966) from \textit{Renewing Inequality} \parencite{nelsonRenewingInequality2025}. 
Urban renewal, established under Title I of the 1949 Housing Act, provided federal subsidies for cities to acquire and clear ``blighted'' neighborhoods. 
Urban renewal was closely intertwined with public housing: clearance projects often created sites for new developments or displaced families that public housing was intended to rehouse \parencite{vonhoffmanStudyContradictionsOrigins2000}.
Recent research finds that Black neighborhoods were two to three times more likely to receive urban renewal projects than white neighborhoods, and that urban renewal areas experienced declines in housing and population density alongside rising rents and incomes \parencite{lavoiceLongrunImplicationsSlum2024}.
To account for potentially confounding effects of urban renewal and to explore the relationship between the two programs, I overlay these maps onto the 1950 Census tract boundaries and classify a tract as an ``urban renewal'' tract if more than 5\% of its area overlaps with an urban renewal project.

Finally, I incorporate data on the locations of interstate highways as of 1996 from \textcite{weiwuUnequalAccessRacial2025}. Interstate highway construction began in the late 1950s, and has been accused of disproportionately displacing minority communities \parencite{roseInterstateHighwayPolitics2012}. These data allow me to control for potential confounding effects of highway construction on neighborhood change, as well as to explore the relationship between highway locations and public housing placement.
\subsection{Defining Treatment and Spillover Neighborhoods}
To study the neighborhood effects of public housing construction, I define a set of \textit{treated} tracts that received public housing projects and a set of \textit{nearby} tracts that may have experienced spillover effects from nearby projects.
To ensure that a project represents a meaningful neighborhood intervention, I include only projects with at least 50 housing units.

Because the public housing projects are geocoded at the coordinate level, I use a geographic buffer approach to define treated tracts.
Specifically, a tract is defined as treated if any portion of it intersects a 100-meter buffer around a public housing project.
This definition captures cases in which a project straddles multiple tracts; when that occurs, I allocate its units and population evenly across the affected tracts.

Since the Census data are decennial, I assign treatment timing by decade.
Following the literature, I use the project's completion date as the treatment date \parencite{asquithLocalEffectsLarge2023}.
A tract is considered treated in year $t$ if its first qualifying project was completed between $t-9$ and $t$ (e.g., tracts receiving projects between 1951 and 1960 are treated in 1960).
This timing convention ensures that the treatment occurs after the pre-treatment observation but before the post-treatment observation in the panel.
\footnote{If a project was completed in the Census year but after the Census enumeration date, the effects of the project would meaningfully be captured in the second, rather than first, post-treatment decade, and the $t=0$ effects would be understated.}
For tracts receiving multiple public housing projects over time, treatment timing is determined by the first project meeting the size thresholds.
Subsequent projects in the same tract are not considered separate treatment events, since the tract has already been ``treated'' by public housing construction.

To examine spatial spillovers, I define a nearby tract as one that shares a border with a treated tract, is not itself treated, and lies within one kilometer of the nearest public housing project.
This hybrid contiguity-and-distance definition identifies neighborhoods that are close enough to plausibly experience externalities from nearby developments while excluding tracts that are technically adjacent to project neighborhoods but geographically distant from a project.
I adopt a one-kilometer threshold based on prior literature showing the geographic extent of spatial spillovers from public housing interventions \parencite{blancoKnockingItMixing2025}.
Nearby tracts inherit the earliest treatment year among their treated neighbors.
Figure \ref{fig:chicago_spillover} illustrates the treated and nearby classification for Chicago, showing treated tracts and their adjacent spillover areas.


\subsection{Sample Selection}
My empirical analysis focuses on projects built between 1941 and 1973, with neighborhood outcomes measured from 1930 to 2010.
I restrict the sample in several ways to ensure a balanced panel of neighborhoods with sufficient pre-treatment data to conduct my difference-in-differences analysis.

Table \ref{tab:sample_attrition} shows the impact of each filtering step on the sample composition.
Starting from the original sample of 12,062 census tracts representing 38.1 million people in 1940 (approximately 29\% of the U.S. population), I apply the following restrictions.
First, I require that tracts exist in all years from 1930 to 2010, which drops about 25\% of tracts. %, primarily due to new tracts created in later decades or areas that lacked tract coverage in early years.
Second, I drop tracts for which I cannot calculate population by race, median income, median rent, labor force participation rates, and unemployment rates for all years, which removes an additional 5\% of tracts.
Third, I exclude tracts with public housing built before 1941 to preserve 1930 and 1940 as clean pre-treatment periods, which removes 109 treated tracts containing 109 projects.
Fourth, I drop metropolitan areas with fewer than 30 total tracts to ensure sufficient within-city variation. This excludes three small metropolitan areas.
Finally, I exclude population outliers: tracts with populations below the 5th percentile or above the 98th percentile in any year.


The resulting balanced sample includes 6,506 census tracts across 47 CBSAs, representing 27.5 million people in 1940 (approximately 21\% of the U.S. population).
This sample contains 814 public housing projects with 300,964 total units located in 822 treated tracts.
%Overall, the filtering process retains 54\% of the original census tracts and 53\% of the original public housing projects matched to census tracts, with the greatest attrition due to balanced panel requirements and population outlier exclusions.
Figure \ref{fig:cbsa_coverage} shows the geographic distribution of CBSAs included in the analysis.
%Table \ref{tab:tract_summary} shows summary statistics for census tract characteristics in 1940 and 1990.

