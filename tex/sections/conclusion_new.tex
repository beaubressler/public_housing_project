\section{Conclusion}
This paper examines the long-run neighborhood effects of one of the most ambitious urban policy interventions in 20th century America: the construction of public housing projects. Using a novel dataset tracking over 1 million public housing units built between 1935 and 1973, combined with 70 years of Census tract-level data, I investigate both where public housing was built and how it transformed American neighborhoods over subsequent decades.

These results are consistent with a story about well-intentioned policy that ultimately reinforced the very problems it sought to address. Public housing site selection followed predictable patterns: projects were significantly more likely to be built in previously redlined neighborhoods with higher Black population shares, lower median incomes, and higher unemployment rates. While this aligned with stated "slum clearance" objectives, it also concentrated public housing in already disadvantaged areas.
The neighborhood effects were profound and persistent. Using a matched difference-in-differences approach, I find that public housing neighborhoods experienced dramatic long-term demographic shifts, with large increases in the Black population share, alongside substantial socioeconomic decline, including large decreases median incomes and high school graduation rates. Crucially, these effects radiated outward: nearby neighborhoods also saw large increases in Black population, declines in median incomes, and  decreases in property values over thirty years. Rather than revitalizing neighborhoods, public housing construction contributed to a "degentrifying" process that reinforced patterns of racial and economic segregation across entire areas of American cities.

Several important limitations should be acknowledged. First, this analysis focuses on neighborhood-level outcomes rather than the experiences of individual public housing residents. While I document substantial effects on neighborhoods, I do not assess whether public housing improved housing quality, affordability, or other outcomes for the families it housed, outcomes that were central to the program's original goals. Second, the matched difference-in-differences approach relies on the parallel trends assumption that matched control neighborhoods provide valid counterfactuals. Although pre-trend tests support this assumption and my matching procedure addresses observable differences, unobserved factors could still influence the results. I am currently working on implementing robustness checks and alternative identification strategies.

In ongoing work, I explore heterogeneity in the effects of the public housing program across project characteristics, cities and time, to better understand whether the negative effects of public housing on neighborhoods can be mitigated by project characteristics. The mechanisms and channels behind the reduced-form results could also be further explored through a neighborhood choice model. I am also in the process of studying the effects of living in public housing projects on individuals' short- and long-run outcomes by linking individuals to the projects themselves.
