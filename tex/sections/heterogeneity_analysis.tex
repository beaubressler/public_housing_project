\section{Heterogeneity Analysis}

I next explore the heterogeneous effects of the neighborhood effects of public housing along three dimensions: project characteristics, initial neighborhood characteristics, and by city. 

Project characteristics may matter in several ways. First, one critique of the projects is that they were large, forboding buildings that dominated their neighborhoods, with some desi projects was a focus of a particular critique of public housing (\cite{newmanCreatingDefensibleSpace1997, jacobsDeathLifeGreat1992}). Indeed, in recent decades, the public housing that is produced tends to be smaller buildings that blends in with local neighborhoods rather than stands out. To get at these dynamics, I explore heterogeneity by project size. The demographics of the project might matter as well, and I explore heterogeneity along those dimensions as well.

There are several reasons to expect that the effects of public housing projects may depend on the characteristics of the neighborhoods in which it was built. In a more modern setting, work by \textcite{diamondWhoWantsAffordable2019} show that the effects of affordable housing construction, in their case, LIHTC construction, depend on the initial characteristics of the neighborhood. They find that new LITHC projects serve as potential positive amenities in lower-income, higher-minority neighborhoods while creating negative externalities in more affluent areas. Furthermore, if racial steering or neighborhood tipping are mechanisms behind the effects on neighborhoods, we might expect that these effects differ depending on the initial racial composition of the neighborhood. 

Finally, I estimate differences across cities. I am particularly interested in whether I estimate different effects in New York City compared to the rest of the country. 

\subsection{Heterogeneity by project characteristics}

I first examine how the effects of public housing vary by project scale. Table X presents pooled difference-in-difference estimates for segregation, population, and economic outcomes by project size, measured by total public housing units in treated tract in the initial treatment year.\footnote{Although several tracts contain multiple projects, I will refer to them jointly as a single project, and let "project size" denote the total units in the treated tract at t=0.} I divide projects into terciles, with small projects (bottom tercile) containing between 56 and 200 units, medium projects containing between 201 and 402 units, and large projects (top tercile) containing more 407 units (up to 4312).
I find 

The results reveal striking threshold effects. Small public housing projects had modest impacts on neighborhood segregation, increasing the local dissimilarity index by X.XX percentage points. However, large projects had dramatically different effects, increasing segregation by an additional X.XX percentage points (Column 3, Panel A). This suggests that project scale fundamentally altered the nature of public housing's neighborhood impacts.
The mechanism behind these differential segregation effects becomes clear when examining population responses in Panel B. While small projects caused modest private population decline (X.XX log points), large projects triggered massive private market flight, with an additional X.XX log point decline relative to small projects. This represents a difference of approximately X,XXX fewer private residents in large project neighborhoods compared to small project areas.
Importantly, the differential effects were not simply due to different numbers of public housing residents. Large projects did attract more Black residents (Column 3, Panel B), but the magnitude of private population decline far exceeded the direct demographic impact of the projects themselves. This suggests that large projects created displacement effects that went beyond their direct occupancy.
The economic consequences followed a similar pattern. Large projects experienced substantially larger declines in median income and rental prices compared to small projects (Panel C). The rent decline was particularly pronounced, with large projects seeing an additional $XX decline relative to small projects, suggesting that private housing markets viewed large projects as more problematic amenities.
These findings align with contemporary critiques of large-scale public housing developments and provide empirical support for the shift toward smaller, scattered-site development in modern housing policy. The results suggest there may be critical thresholds in public housing scale above which neighborhood effects become qualitatively different—characterized by demographic replacement rather than accommodation.RetryClaude can make mistakes. Please double-check responses.