\section{Chicago High-Resolution Land Value Analysis}
\label{app:chicago_land_values}

This appendix presents a supplementary spatial analysis of the effects of public housing on land values in Chicago using a high-resolution panel of land values.
I use a dataset from \textcite{ahlfeldtTallBuildingsLand2018} that provides nearly-decadal data on land values at a 300×300 foot grid-cell level from 1913 to 2010, sourced from \textit{Olcott's Land Values - Blue Book of Chicago}.
While my baseline analysis estimates the effects of public housing on Census self-reported rents, assessed land values may provide additional insights into the economic impacts of these projects.
Furthermore, the spatial granularity of these data is particularly well-suited for a spatial difference-in-differences design using concentric rings around project locations.

\subsection{Empirical Strategy}

I estimate the effects of public housing on land values using a stacked spatial difference-in-differences design following \textcite{blancoKnockingItMixing2025}, leveraging the high-resolution grid-cell data to analyze how public housing projects affected land values in concentric rings around project locations.
As discussed in Section \ref{sec:site_selection}, the locations of public housing projects were determined by a combination of neighborhood characteristics.
Here, given the spatial granularity of the data, I define the comparison points based on proximity to public housing project locations.

For each project, I compare the evolution of land values in a ring of a certain radius around each project to land values in a control ring further away.
I define treatment rings as 200m concentric buffers around each public housing project, and define the control rings as the ring of grid points 800-1000m away from the project.
The identifying assumption is that the trends in land values would have been similar between the treatment rings (0-800m) and the control rings (800-1000m) in the absence of public housing construction.
The data and estimation is organized in a "stacked" difference-in-differences framework, with each public housing project treated as a separate "sub-experiment" (\cite{wingStackedDifferenceDifferences2024}).

I estimate the following event study at the grid cell $i$, public housing project $g$, and year $t$ level: 
\begin{align}
\text{llv}_{igt} = \sum_{k \neq -1} \sum_{r \in R} \beta_{kr} \mathbf{1}[\text{event\_time}_{it} = k] \times \mathbf{1}[\text{Ring}_{ig} = r] + \gamma_{igt} + \epsilon_{igt}
\end{align}

where $\text{llv}_{igt}$ is the log land value in grid cell $i$ at time $t$ around public housing project $g$.
%TODO: BEAU start here
$\text{event\_time}_{it}$ is the event time relative to the project opening, and $\text{Ring}_{ig}$ indicates the spatial ring type (e.g., 0-200m, 200-400m, etc.).
The coefficients $\beta_{kr}$ capture the effects of public housing projects on land values at different event times $k$ and across different spatial rings $r$.

where $i$ indexes grid cells, $g$ indexes projects, $t$ indexes years, $k$ indexes event time, and $r$ indexes spatial rings. The fixed effects $\gamma_{igt}$ include:

\begin{itemize}
    \item \textbf{Project×Year}: $\alpha_{gt}$ controls for project-specific time trends
    \item \textbf{Project×Ring}: $\mu_{gr}$ controls for baseline differences across spatial rings within each project
    \item \textbf{Project×Urban Renewal}: $\delta_{g} \times \text{UR}_{ig}$ allows heterogeneous effects by urban renewal exposure  
    \item \textbf{Project×Highway Proximity}: $\theta_{g} \times \text{Highway}_{ig}$ controls for transportation infrastructure
\end{itemize}

The \textbf{Project×Location Type} fixed effects ($\mu_{gr}$) ensures that identification comes from within-project spatial-temporal variation rather than cross-project comparisons.


\subsection{Results}

Figure \ref{fig:chicago_event_study_main} presents the main results from the spatial land value analysis.
The baseline specification shows that public housing projects, on average, had positive effects on local land values, particularly in the immediate vicinity of the projects.
These effects mostly do not reach statistical significance, however. 


  \begin{figure}[htbp]
      \centering
      \includegraphics[width=0.8\textwidth]{../output/figures/chicago_land_values/chicago_land_value_event_study_all_rings.pdf}
      \caption{Effect of Public Housing on Chicago Land Values - Main Event Study}
      \label{fig:chicago_event_study_main}
  \end{figure}


Figure \ref{fig:chicago_hetero_size_median} examines how these effects vary based on the size of the projects. 
I estimate the main specification separately for projects above the median project size (202 units) and those equal or below, comparing 55 smaller projects to 18 larger developments. 
This analysis reveals substantial heterogeneity in neighborhood land value impacts.
In particular, I find some evidence that smaller projects seemed to have more persistent positive effects on local land values.  

  \begin{figure}[htbp]
      \centering
      \includegraphics[width=0.9\textwidth]{../output/figures/chicago_land_values/chicago_land_value_heterogeneity_median_split.pdf}
      \caption{Heterogeneity by Project Size - Median Split Event Study}
      \label{fig:chicago_hetero_size_median}
  \end{figure}



