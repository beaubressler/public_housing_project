\section{Chicago High-Resolution Land Value Analysis}
\label{app:chicago_land_values}

This appendix presents a supplementary spatial analysis of the effects of public housing on land values in Chicago using a high-resolution panel of land values.
I use a dataset from \textcite{ahlfeldtTallBuildingsLand2018} that provides nearly-decadal data on land values at a 300×300 foot grid-cell level from 1913 to 2010, sourced from \textit{Olcott's Land Values - Blue Book of Chicago}.
While my baseline analysis estimates the effects of public housing on Census self-reported rents, assessed land values may provide additional insights into the economic impacts of these projects.
Furthermore, the spatial granularity of these data is particularly well-suited for a spatial difference-in-differences design using concentric rings around project locations.

\subsection{Empirical Strategy}

I estimate the effects of public housing on land values using a stacked spatial difference-in-differences design following \textcite{blancoKnockingItMixing2025}, leveraging the high-resolution grid-cell data to analyze how public housing projects affected land values in concentric rings around project locations.
As discussed in Section \ref{sec:site_selection}, the locations of public housing projects were determined by a combination of neighborhood characteristics.
Here, given the spatial granularity of the data, I define the comparison points based on proximity to public housing project locations.

For each project, I compare the evolution of land values within a ring of a given radius around each project to those in a control ring farther away.
I define treatment rings as 200m concentric buffers around each public housing project, and define the control rings as the ring of grid points 800-1000m away from the project.
The data and estimation are organized in a stacked difference-in-differences framework with each public housing project treated as a separate ``sub-experiment'' (\cite{wingStackedDifferenceDifferences2024}).
The identifying assumption is that the trends in land values would have been similar between the treatment rings (0-800m) and the control rings (800-1000m) around each project in the absence of public housing construction.

I estimate the following event study at the grid cell $i$, public housing project $g$, and year $t$ level:
\begin{align}
\text{llv}_{igt} = \sum_{k \neq -1} \sum_{r \in R} \beta_{kr} \mathbf{1}[\text{event\_time}_{gt} = k] \times \mathbf{1}[\text{Ring}_{ig} = r] + \alpha_{gt} + \mu_{gr} + \gamma_{igt} + \epsilon_{igt}
\end{align}

where $i$ indexes grid cells, $g$ indexes public housing projects, $t$ indexes years, $k$ indexes event time, and $r$ indexes spatial rings.
The variable $\text{llv}_{igt}$ is the log land value in grid cell $i$ at time $t$ around public housing project $g$, $\text{event\_time}_{gt}$ is the event time relative to project $g$'s opening, and $\text{Ring}_{ig}$ indicates the spatial ring type (e.g., 0-200m, 200-400m, etc.).
The coefficients $\beta_{kr}$ capture the effects of public housing projects on land values at different event times $k$ and across different spatial rings $r$.
The term $\alpha_{gt}$ represents project-by-year fixed effects that ensure identification comes from comparing grid cells within each project over time, rather than across different projects.
The term $\mu_{gr}$ represents project-by-ring fixed effects that control for baseline differences across spatial rings within each project.
The term $\gamma_{igt}$ includes additional controls---urban renewal status and proximity to an interstate highway---interacted with project identifiers to allow effects to vary flexibly within each project.


\subsection{Results}

Figure \ref{fig:chicago_event_study_main} presents the main results from the spatial land value analysis.
The baseline specification shows that public housing projects, on average, had positive effects on local land values, particularly in the immediate vicinity of the projects.
These effects mostly do not reach statistical significance, however. 


  \begin{figure}[htbp]
      \centering
      \includegraphics[width=0.8\textwidth]{../output/figures/chicago_land_values/chicago_land_value_event_study_all_rings.pdf}
      \caption{Effect of Public Housing on Chicago Land Values - Main Event Study}
      \label{fig:chicago_event_study_main}

      \small
      \note{Event study estimates of the effects of public housing construction on log land values in Chicago, using a stacked spatial difference-in-differences design. Estimates are shown for 200-meter concentric rings around public housing projects (0-200m, 200-400m, 400-600m, 600-800m), with grid cells 800-1000m from projects serving as the control ring. The reference period is one period before construction (event time = -1). Vertical bars represent 95\% confidence intervals with standard errors clustered at the project level. Source: Land value data from \textcite{ahlfeldtTallBuildingsLand2018}.}
  \end{figure}


Figure \ref{fig:chicago_hetero_size_median} examines how these effects vary based on the size of the projects. 
I estimate the main specification separately for projects above the median project size (202 units) and those at or below it, comparing 55 smaller projects to 18 larger developments. 
This analysis reveals substantial heterogeneity in the impacts of neighborhood land values.
In particular, I find some evidence that smaller projects seemed to have more persistent positive effects on local land values.  

  \begin{figure}[htbp]
      \centering
      \includegraphics[width=0.9\textwidth]{../output/figures/chicago_land_values/chicago_land_value_heterogeneity_median_split.pdf}
      \caption{Heterogeneity by Project Size - Median Split Event Study}
      \label{fig:chicago_hetero_size_median}

      \small
      \note{Event study estimates comparing the effects of smaller ($\leq$202 units, N=55 projects) and larger ($>$202 units, N=18 projects) public housing projects on log land values in Chicago. Estimates are shown separately for each project size category across the same concentric rings as in Figure \ref{fig:chicago_event_study_main}. Vertical bars represent 95\% confidence intervals with standard errors clustered at the project level. Source: Land value data from \textcite{ahlfeldtTallBuildingsLand2018}.}
  \end{figure}



